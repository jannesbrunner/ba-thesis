\section{Einführung}\label{sec:einfuhrung}
\subsection{Motivation}\label{sec:motivation}

Am 04.04.2019 trat die Änderung des Art. 104c des Grundgesetz für die Bundesrepublik Deutschland in Kraft
und ebnete so den Weg für den von Bund und Ländern beschlossenen Digitalpakt Schule \cite{Art104cG55:online}. 
Dieser Beschluss macht deutlich, dass digitale Kompetenz im Bildungssektor von hoher Bedeutung ist, was auch von einer Förderungssumme von mindestens 5,5 Milliarden Euro unterstrichen wird. 
Legt man diese Summe auf die ca 40.000 Schulen um, erhält jede Schule einen Durchschnittsbeitrag von 137.000 Euro. Bei ca. 11 Millionen Schülerinnen und Schülern würde das eine Förderungssumme von ca. 500 Euro pro Schüler bedeuten. 
Einer der Hauptförderungspunkte des Digitalpakt Schule sieht den Ausbau der technischen Infrastruktur
an deutschen Schulen vor, z.B. Bereitstellung von drahtlosen Netzwerken, schnellen Internetzugangspunkten und digitale Unterrichtsmedien wie interaktive Whiteboards.
\\ \\
Das Bundesministerium für Bildung und Forschung (BMBF) argumentiert jedoch auch damit, dass kein digitales Medium alleine gute Bildung fördert, sondern immer dahinterstehende pädagogische Konzepte aus einer Vielfalt von Angeboten entscheidend sind \cite{dpakt2019:online} Dennis Horn (Experte für digitale Themen der ARD) kritisiert den zu starken Fokus auf Hardware und mahnt an, dass zu wenig darüber gesprochen wurde, wie diese denn auch sinnvoll genutzt werden kann. Man bereite keine Schülerinnen und Schüler auf eine digitale Welt vor, allein durch das Verteilen von Tablet-Computern\cite{Horn2018:online} \\ \\
Diese Problematik wurde auch auf der Podiumsdiskussion der re:publica 2018 - 'Was kommt in den digitalen Schulranzen?' angeschnitten. Tobias Hübner, Lehrer und Autor im Bereich Medienistik, zeigt dort ebenfalls auf, dass der Wille Geld auszugeben zu begrüßen sei, es aber an Konzepten und Materialien mangele. Als Lehrer würde er den Investitionsfokus auf Lehrerfortbildung setzen.
Es bestünde bereits eine Grundbereitschaft seitens vieler Lehrenden kleiner Ausgaben im Bereich Digitales selbst zu tätigen. Als ein Beispiel wäre hier der Einplatinencomputer Raspberry Pi zu nennen, welcher bereits für 33 Euro erwerblich ist (Stand April 2019) und genug Rechenkapazitäten bereitstelle um zahlreiche Projekte im Bildungsbereich durchzuführen. 
\\ \\
Im Vergleich kostet der populäre Tablet Computer 'iPad' der Firma Apple inc. in der günstigsten Variante bereits mindestens 449€ \cite{iPadmini65:online} (Stand April 2019), was schon knapp 90\% des Förderungsvolumens pro Schülerin und Schüler ausmachen würde.
\\ 

Seit dem Erfolgskurs des Web 2.0\footnote{Web 2.0 ist ein Schlagwort, das für eine Reihe interaktiver und kollaborativer Elemente des Internets, speziell des World Wide Webs, verwendet wird. Dabei konsumiert der Nutzer nicht nur den Inhalt, er stellt als Prosument selbst Inhalte zur Verfügung. - Wikipedia.org} in den frühen 2000er Jahren, zeichnet sich auch zunehmend der Trend des Software-as-a-Service Geschäftsmodells ab. Dies beschreibt die Bereitstellung von Software im Internet, ohne dass der Benutzende die Software selbst noch lokal installiert haben muss. Im Jahr 2015 setzten bereits über drei Viertel von 102 befragten Unternehmen Software dieser Form aktiv im Geschäft ein\cite{TecArt-GmbH2019:online} Dieses Phänomen, oft auch digitale Transformation betitelt, wurde nicht zuletzt durch die immer leistungsstärker werdenden Web-Technologien möglich. Viele Arten von Software können  mittlerweile in einer im Webbrowser lauffähigen Alternative substituiert werden. Ein populäres Beispiel ist die Office-Suite Google Docs der Firma Google inc. Hier lassen sich Textverarbeitung, Tabellenkalkulation und das erstellen von Präsentationen ohne Installation und direkt im Webbrowser des Benutzenden ausführen. Ein anderes Beispiel ist die Web-Software Photopea welche ebenfalls komplett im Web-Browser ausgeführt wird und dem nur lokal installiert ausführbaren quasi Industriestandard Bildbearbeitungsprogramm Photoshop der Firma Adobe inc. sehr nahe kommt. Im Vergleich zu lokal installierter Software ist die Bereitstellung von Web-Software einfacher, da solange ein moderner Webbrowser lauffähig ist, das Betriebssystem des Client-Computers zu vernachlässigen ist. Ebenso stellt potente Hardware keine zwingende Voraussetzungen, da etwaige rechenintensive Aufgaben auf der Serverseite getätigt werden können oder hier eine Balance zwischen Client und Server angestrebt werden kann. \\ 

Da ein o.g. Raspberry Pi Einplantinencomputer bereits genügend Leistung für Webtechnologien aufweist und einen sehr günstigen Anschaffungspreis aufweist, sowie bereits 67\% der 10-11 jährigen Jugendlichen ein Smartphone besitzen\cite{Statista2017:online} welche ebenfalls genug Leistung für Webanwendungen aufweisen, könnte der verstärkte Einsatz von Software, welche auf Webtechnologien basiert, die im Rahmen des Digitalpakt Schule fließenden Gelder optimierter ausschöpfen. 
%Hier jetzt Web Trend noch erwähnen!

\subsubsection{Besuch der Grundschule am Rüdesheimer Platz}
Im Rahmen der Vorrecherche zu dieser Arbeit wurde einem Unterrichtstag in 
einer Jahrgangsübergreifenden (JüL) Klasse 1 bis 3 an der Grundschule am Rüdesheimer Platz beigewohnt um ein differenzentierese 
Bild der gegenwärtigen Lern- und Digitalisierungssituation an einer Berliner Schule zu bekommen. An dieser Stelle eine große Dankaussagung an Frau Marie Wewer, Grundschullehrerin, welche diese Erfahrung möglich gemacht hat und in einem anschließenden Gespräch das Interesse an einer kostengünstigen und einfach nutzbaren Lösung zur Unterstützung von interaktiven Unterrichtsmethoden unterstrichen hat. Die Erprobung der im Rahmen dieser Arbeit implementierten Softwarelösung wurde ebenfalls an der Grundschule am Rüdesheimer Platz durchgeführt und wird im Kapitel \ref{sec:erprobung} erläutert.     
