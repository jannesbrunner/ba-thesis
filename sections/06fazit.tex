\section{Auswertung}\label{sec:auswertung}
In diesem vorletzten Kapitel dieser Ausarbeitung soll die implementierte Software als Endresultat mit den ursprünglichen Plänen und Vorstellungen verglichen werden. Ebenso wird über einzelne Bestandteile des Projektes hinsichtlich Optimierung gesondert diskutiert und anschließend gesammelte Erfahrungen, die mit der Umsetzung des Projekts einhergingen, reflektiert. 
Im letzten Abschnitt \textbf{\ref{sec:ausblick} Ausblick} wird auf die Zukunft des Projektes näher eingegangen.
\\ \\
\subsubsection{Umsetzung versus Planung}\label{sec:umvsplan}
Im Rückblick auf die ersten Vorstellungen und Entwürfe hinsichtlich des Projekts, wurden diese im Bezug auf Grundfunktionalität und Design erfüllt. Die entwickelte Serversoftware ist flexibel auf verschiedenen Betriebssystemen und unterschiedlicher Netzwerk-Infrastruktur einsetzbar. Dabei ist ein Offline im Intranet gänzlich möglich und somit keine Ressourcen, die aus dem Internet geladen werden müssen, notwendig. Lehrende oder Verwalter können die Software mit wenig Aufwand installieren und bis auf die NodeJS Umgebung mitsamt NPM ist kein tiefgreifender Systemeingriff notwendig. Neue Nutzer können sich einfach nach der Installation registrieren, Lehreinheiten anlegen und diese ausführen. Die zwei Unterrichtsmethoden Brainstorming und Quiz wurden erfolgreich umgesetzt. Die angestrebte Drei-Client Lösung wurde implementiert und die Clients können je nach Situation und Anforderung auf unterschiedlichen Geräten ausgeführt werden. Die Anwendung läuft stabil und weist nur selten Fehler auf. Insgesamt kann der Code jedoch an vielen Stellen noch mittels Refactoring optimiert werden. 
\subsubsection{Verbesserungsvorschläge}\label{sec:verbesserung}
Während des Entwicklungsprozesses sind einige Problemstellen zum Vorschein getreten, die im Nachhinein Verbesserungspotential hinsichtlich Usability, Performanz oder Software-Design aufweisen. \\ \\
Der Nachrichtenaustausch welcher über das WebSocket Protokoll via SocketIO realisiert ist, sollten mehr vereinheitlicht werden. Grundsätzlich lassen sich jede Art von JavaScript Daten übertragen, ein strengeres Konzept kann hier das Verständnis für andere Entwickler fördern und den Code sauberer gestalten. Gerade der Ausführungscode der interaktiven Unterrichtsmethoden könnte noch besser gekapselt und noch sinnvoller aufgeteilt werden, auch in Hinsicht der Daten, welche zwischen Server und Client ausgetauscht werden. Der Installationsprozess könnte ggf. noch automatisierter erfolgen und so wenig Technik affinen Nutzenden die Installation erleichtern.  
Wie im vorangegangen Abschnitt bereits erwähnt, kann der Code der Clients, welche im Webbrowser ausgeführt wird, mit mehr Kenntnissen über die Entwicklung mit VueJS und SocketIO maßgeblich optimiert werden. \\ \\ Zum Zeitpunkt des Abschlusses des Projekts kann eine Lehrkraft ein Lehreinheit anlegen, welche eine Unterrichtsmethode (in diesem Falle Brainstorming oder Quiz) beinhalten kann. Eine Verkettung von mehreren Unterrichtsmethoden wäre wünschenswert (Set Betrieb, siehe auch Abschnitt \ref{sec:gegenstellung}). 


\subsubsection{Erfahrungsauswertung}\label{sec:erfahrungen}
Aufgrund der Arbeit an diesem Projekt wurden vielerlei Erfahrungen hinsichtlich der Entwicklung eines verteilten Systems in Form eine Webanwendung gesammelt. Insbesondere die gewonnen Kenntnisse im Umgang mit den Frameworks NodeJS, Express und SocketIO waren lehrreich und der erlangte Wissensstand kann als Basiswissen hinsichtlich vielerlei Arten von Projekten in Zukunft genutzt werden. Auf der Client Seite war der Einblick in die Arbeitsweise von VueJS interessant. Das Vorwissen über die Konkurrenten Angular und React war hilfreich wurde jedoch definitiv um neue Ansätze ergänzt. Die in der neusten Iteration des JavaScript hinzugefügten Funktionalitäten wurde verinnerlicht und das Wissen um die Sprache intensiviert.     

\subsubsection{Ausblick}\label{sec:ausblick}
%electron!
%  Eine hier angesetzte Refaktorierung ist vuejs und socketIO
% REST statt Sockets
% 
Nach Beendigung des Bachelorprojekts soll dieses nicht verworfen werden sondern weiterhin privat ausgebaut werden. Während der Entwicklung kamen immer wieder Ideen für weitere Funktionalitäten auf, welche aber aus zeitlichen Gründen nicht implementiert worden sind oder nur als Konzept vorlagen. Dies wären z.B. Funktionen die den Dozent noch weiter während des Unterrichts unterstützen könnten oder das schreiben einer API, um die Software an andere Systeme anbinden zu können. Ein mögliches Vorhaben wäre es, die Software in einem Fork nach dem REST-Design aufzubauen und die Kommunikation anders zu gestalten, um anschließend zu vergleichen, welche Vorgehensweise entsprechende Vor- und Nachteile mit sich bringt. Auch könnte Dank der NodeJS Basis des Projektes sich unabhängig eine 'richtige' Desktop Applikation entwickelt werden, was mit dem Electron Framework realisierbar wäre.  
