\section*{Abstract}\label{sec:abstract}
Im Zuge der Digitalisierung sind Bildungseinrichtungen mit  
neuartigen nie dagewesen Problemen konfrontiert. 
Wie können bewährte und neue Unterrichtsmethoden sinnvoll durch digitale
Technik unterstützt werden und der finanzielle Rahmen der zur Verfügung stehenden
Mittel, gerade im Zuge des Digitalpakts Schule optimal genutzt werden? Ist Datenschutz gewährleistet? 
\\ \\
Viele Anbieter setzen auf Cloudlösungen, welche eine Internetanbindung zur Nutzung obligatorisch macht. Schulen mit noch ausbaufähiger digitaler Infrastruktur 
bedarf es jedoch an skalierbaren Hard- und Softwarelösungen, die zur Not auch offline einsetzbar sind, um auch jetzt schon kosteneffizient digitale Technik im Unterrichtsalltag zu integrieren.
\\ \\  
Im Rahmen dieser Arbeit soll die gegenwärtige Situation in Sachen Digitalisierung hinsichtlich Soft- und Hardware sowie dem damit verbundenen Einsatz von Softwareanwendungen zu Unterstützung von interaktiven Unterrichtsmethoden analysiert werden. Anschließend soll darauf aufbauend eine Client-Server Web-Anwendung zur Lösung implementiert werden, welche in erster Linie offline funktioniert, wenig Anforderungen an die technische Infrastruktur stellt, aber sich auch in ausgebauter Umgebung sinnvoll nutzen lässt. 