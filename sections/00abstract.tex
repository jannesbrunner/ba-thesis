\section*{Abstract}\label{sec:abstract}
Im Zuge der Digitalisierung sind Bildungseinrichtungen mit  
neuartigen, nie dagewesen Problemen konfrontiert. 
Wie können sowohl bereits bewährte, als auch neue Unterrichtsmethoden sinnvoll durch digitale
Technik unterstützt werden? Wie kann der finanzielle Rahmen von zur Verfügung stehenden
Mittel optimal genutzt werden? Ist Datenschutz gewährleistet? 
\\ \\
Viele Anbieter setzen auf Cloud-Lösungen, welche eine Internetanbindung zur Nutzung obligatorisch macht. Schulen mit noch ausbaufähiger digitaler Infrastruktur 
benötigen jedoch skalierbaren Hard- und Softwarelösungen, die zur Not auch offline einsetzbar sind, um auch zeitnah kosteneffizient digitale Technik im Unterrichtsalltag zu integrieren.
\\ \\  
Im Rahmen dieser Arbeit wurde die gegenwärtige Situation in Sachen Digitalisierung hinsichtlich Soft- und Hardware sowie dem damit verbundenen Einsatz von Softwareanwendungen zu Unterstützung von interaktiven Unterrichtsmethoden analysiert. Anschließend darauf aufbauend ist eine Client-Server Web-Anwendung zur Lösung implementiert worden, welche in erster Linie offline funktioniert, wenig Anforderungen an die technische Infrastruktur stellt, aber sich auch in ausgebauter Umgebung sinnvoll nutzen lässt. 