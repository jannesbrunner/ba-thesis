\section{Zielsetzung}\label{sec:zielsetzung}
%wichtigste Quellen hier noch nennen un
% - Hypothese formulieren  
% - Methode (Arbeitsschritte/Vorgehensweise) + wichtigste Quellen benennen 

Das Ziel der Arbeit ist es, eine auf Webtechnologien basierende Softwarelösung zur Unterstützung von interaktiven Unterrichtsmethoden zu entwickeln.

Dabei soll auch untersucht werden, ob eine gänzlich vom Internet unabhängige Lösung umsetzbar ist. Sollte keine ausreichende Netzwerkinfrastruktur vorhanden sein, soll diese von der Software selbst in Form eines drahtlosen Netzwerks eingerichtet und gestellt werden. Dies könnte 
Schulen mit langsamer oder gar nicht vorhandene Internetanbindung und Netzwerkinfrastruktur den Einsatz von 
digital gestützten Unterrichtsmethoden erleichtern. 
\\ \\
Darauf aufbauend sollen die Mindesthardwareanforderungen gering gehalten werden. 
Ein niedriger Einrichtungspreis könnte den Rahmen der im \emph{Digitalpakt Schule} fließenden Gelder effizienter ausschöpfen und Schulen mehr finanzielle Flexibilität einräumen. 
Ebenso soll den Nutzenden der Software die Handhabung einfach gemacht werden, um auch technisch weniger Versierten die Nutzung zu ermöglichen. Die zu entwickelnde Software soll Benutzerkomponenten für Lehrende sowie Schülerinnen und Schüler bieten. Darüber hinaus soll eine
optional nutzbare angepasste Darstellungsoption der Software für Klassenräumen implementiert werden, wie z.B. Projektoren, Fernseher u.Ä. Lehrenden soll damit die Möglichkeit gegeben werden, vorhandene Hardware im Klassenraum flexibel einsetzen zu können. Die soll Schülerinnen und Schüler eine visuell ergonomische Nutzung ermöglichen. 
Im Rahmen der Entwicklung soll generell auf gute Gebrauchstauglichkeit (\emph{Usability}) und ein gutes Nutzungserlebnis (\emph{User Experience}) geachtet werden.  
\\ \\
Diese Arbeit widmet sich in diesem Zusammenhang der Umsetzung von Software, welche diese Unterrichtsmethoden abbildet. Damit verbundene pädagogische Aspekte werden nur im Rahmen der Softwareentwicklung beleuchtet.
\\
Der Begriff Unterrichtsmethoden meint in dieser Arbeit Lerninhalte, welche sich digital gestützt im Rahmen einer Unterrichtssituation von Dozierenden mit Teilnehmenden interaktiv durchführen lassen, wie z.B. Quiz-Anwendungen, Brainstorming, Memory-Spiele etc. Weiterführend hierzu sei an dieser Stelle der
Methodenpool von \textit{Netzwerk Digitale Bildung} \cite{NDB} genannt, welcher unter folgender Webadresse erreichbar ist: \\
\url{www.netzwerk-digitale-bildung.de/information/methoden/methodenpool/}
\\ \\
