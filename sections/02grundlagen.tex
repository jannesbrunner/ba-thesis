\section{Grundlagen}\label{sec:grundlagen}
\subsection{Digitalisierung an Schulen}
\subsubsection{Digitale Technik im Unterricht}\label{sec:technikunterricht}
\subsubsection{Ausblick Interaktive Unterrichtsmethoden}\label{sec:interaktiveunterr}
\subsubsection{Datenschutz an Schulen}\label{sec:datenschutz}

\subsection{Überblick Webtechnologie}\label{sec:webbasedsoftware}
% Hier auf PDF Technische Anforderungen verweisen (Footnote?) da sehr ausführlich und gut
% Anfang Geschichtlich
Diese Sektion soll einen grundlegenden über im Kontext dieser Arbeit wichtigen Begrifflichkeiten bieten. Die folgenden Untersektion 2.2.1 ff. werden die Thematiken nur grob umreißen, da eine detaillierte Betrachtung der genannten Begriffe den Rahmen dieser Arbeit weit überschreiten würde. 

\subsubsection{Intranet und Internet}\label{sec:intranetundinternet}
% Detailgrad so sinnvoll?
% Unterschied und Gemeinsamkeit klar machen
% Hier kommunikation erklären! Protokolle und OSI schicht!
Einfach ausgedrückt, ist das Internet ein Netzwerk von Computern, welche weltweit miteinander vernetzt sind. Seine Anfänge lassen sich auf das Ende der 1960er in den USA datieren, als die DARPA (Defense Advanced Research Projects Agency) eine weltweite Verknüpfung von Datennetzen anstrebte. Das hier draus resultierende ARPANET (Advanced Research Projects) kann als Ursprung angesehen werden. Dabei beschreibt der Begriff Internet streng genommen ein 'interconnected network', also ein international vernetztes Netzwerk, ohne dabei die Hardware- und Netzwerktechnologie genauer zu beschreiben \cite{safran2011webtechnologien}.  \\ 
Der wohl populärste Anwendung des Internets ist das World Wide Web, welche gegen das Jahr 1989 von einer Forschungsgruppe rund um Sir Tim Berners-Lee ins Leben gerufen wurde und heute oftmals als Synonym für das gesamte Internet sprachlich genutzt wird. \\ 

In unser heutigen globalisierten Welt lässt sich das Internet mitsamt World Wide Web nicht mehr wegdenken und ist ein integraler Bestandteil der Informationskultur. 
 \\ \\
 Das Intranet beschreibt analog dazu ein lokal abgeschlossenes Netzwerk von Computern, bspw. innerhalb eines Unternehmens. Dabei endet ein Intranet klar an seinen Grenzen und ein Gateway fungiert als Übergabepunkt ins Internet. Die Vernetzung der Endgeräte erfolgt kabelgebunden (LAN) oder kabellos (WLAN). Die Kommunikationsgeschwindigkeit innerhalb eines Intranets sind i.d.R. deutlich höher als im Internet, da Daten nicht erst nach außen an einen Internet Service Provider übermittelt werden müssen. Ein Intranet funktioniert unabhängig vom öffentlichem Internet (erhöhte Ausfallsicherheit), ist nicht öffentlich zugänglich und bietet oft andere oder zusätzliche Funktionen. \cite{Intranet62:online}. 
 
 \subsubsection{Client-Server Modell}\label{sec:clientservermodell}
 Das Client-Server Modell beschreibt das Prinzip der Kommunikation zwischen zwei Teilnehmer innerhalb eines Netzwerks. Grundlegend unterscheidet das Modell hierbei zwischen einer Anbieterseite (Server) und einer Benutzerseite (Client). Der Client betreibt auf seinem Endgerät (Computer, Smartphone, etc.) eine Clientsoftware mit der die Verbindung zum Server aufgebaut wird. Im Fall des WWW (siehe \ref{sec:www}) ist dies in den meisten Anwendungsszenarien ein Webbrowser. Der Client fordert dabei eine Resource an, welche auf dem Server vorliegt oder dort speziell für die Anfrage des Clients generiert wird (siehe auch Sektion \ref{sec:webanwenservices}). Das Client-Server Modell sieht vor, dass immer der Client die Verbindung aufbaut, nie andersherum \cite{ElektronikKompendium.de}. Die Anfrage des Clients wird Request genannt, die Antwort des Servers Response oder Reply, welche bei ausreichender Berechtigung des Clients auch Daten enthält. 
 Server-Computer sollen rund um die Uhr erreichbar sein, während Client-Endgeräte auch abgeschaltet werden können, ohne die Integrität des Netzwerks zu beeinflussen. 
  % Vergleich zu anderen Modellen?
 
\subsubsection{Kommunikation}\label{sec:kommunikation}
Die Kommunikation im Internet und Intranet erfolgt über Protokolle. 
Ein Protokoll kann als ein Satz von Kommunikationsregelvorschriften verstanden werden \cite{safran2011webtechnologien}, welche den Netzwerkverkehr auf unterschiedlichen Schichten reglementieren. 
Diese Schichten werden im OSI-Modell (Open System Interconnection) der ISO (International Standardization Organisation), der internationalen Standardisierungsorganisation beschrieben. (Siehe Tabelle) %HIER TABELLE!
\\ 
Das OSI-Modell ist dabei in sieben Schichten eingeteilt, während die Erste als physikalische Schicht definiert ist und die Siebte als Anwendungsschicht. Protokolle sind dabei jeweils nur über Protokolle benachbarter Schichten in Kenntnis gesetzt. Das OSI-Modell lässt sich grob in anwendungsorientierte Schichten (1 bis 4) und transportorientierte Schichten (5 bis 7) unterteilen. Die im Rahmen dieser Arbeit genutzten Webtechnologien nutzen kommunikativ nur anwendungsorientierte Schichten des ISO-OSI Modells.



% Begriff Internet
\subsubsection{World Wide Web}\label{sec:www}
Das World Wide Web (WWW) ist die wohl populärste Anwendung des Internets \cite{safran2011webtechnologien} und wird oftmals fälschlicherweise als Synonym für das gesamte Internet genannt. Das WWW ist eine Sammlung von verteilten Dokumenten, welche sich gegenseitig über sog. Hyperlinks referenzieren und von Web-Servern zur Verfügung gestellt werden. Auf der Client Seite (siehe \ref{sec:clientservermodell}) stellt der Web-Browser die wichtigste Software da. Mit ihr werden Web Server angesprochen (Request) und Antworten (Response) für den Nutzenden dargestellt. Die wichtigsten sprachlichen Komponenten des WWW sind: \\ 
\begin{itemize}
	\item HTML: Hypertext Markup Language - eine reine Beschreibungssprache, welche Hypertext Dokumente durch Tags codiert. 
	\item CSS: Cascading Style Sheet - Eine Stylesheet Sprache, welche das äußere Erscheinungsbild von Hypertext Dokumenten beschreibt
	\item JS: JavaScript: Eine Skriptsprache, welche u.A. Interaktion sowie Dynamik hinzufügt und clientseitig interpretiert wird. 
\end{itemize}
Die Techniken des WWW können auch lokal im Intranet genutzt werden. 
Das zur Verständigung zwischen Client und Server genutzte Protokoll (siehe \ref{sec:kommunikation})ist das Hypertext Transfer Protocol (HTTP) bzw. in verschlüsselter Form Hypertext Transfer Protocol Secure, da eine Übermittlung im Klartext nicht immer wünschenswert ist. HTTP/HTTPS ist ein Zustandsloses Protokoll, das bedeutet dass jede Anfrage unabhängig voneinander geschieht und betrachtet wird. Dies und die Tatsache, dass jede Anfrage von der Client-Seite aus gestartet werden muss (siehe \ref{sec:clientservermodell}), stellen oftmals Hürden für die Entwicklung von Webanwendungen und Webservices da. Techniken wie Cookies und Sessions, sowie das wiederholte Abfragen von aktualisierten Daten seitens des Clients wirken hier entgegen. Cookies stellen persistent gespeicherte Daten auf der Client-Seite da, mit deren Hilfe der Webserver einen Client eindeutig zuordnen kann. Bei einer Session sendet der Client bei jeder Anfrage eine eindeutige ID an den Server. Im Normalfall endet eine Session beim Beenden des Webbrowser, während Cookie-Dateien eine längere Lebensdauer besitzen.      
%
\subsubsection{Webanwendungen und Webservices}\label{sec:webanwenservices}
Im Laufe der Entwicklung des WWW (\ref{sec:www}) stieg der Anspruch vom reinen Anbieten statischer Dokumenten in Richtung dynamischer Inhalte, welche einer Programmlogik folgend von einem Webserver für jede Anfrage generiert werden. Webanwendungen sind Computerprogramme, welche auf einem Webserver ausgeführt werden und den Webbrowser des Clients als Schnittstelle nutzen \cite{safran2011webtechnologien}. Dies bietet den großen Vorteil, dass etwaige Anpassungen von Programmlogik nur serverseitig erfolgen müssen und jeder Client mit Webbrowser als Benutzerschnittstelle ausreicht. \\ \\
Webservices sind eine spezialisierte Art von Webanwendung. Die Fokus hier liegt auf das bereitstellen von Daten für andere Applikationen, welche die gewonnen Daten selbst auswerten und dem Nutzenden bereitstellen. Dies geschieht i.d.R. über eine einheitlich beschriebene Schnittstelle (API - Application Programming Interface), über welche fremde Applikationen angefragte Daten abrufen können. Der Austausch der Daten erfolgt hier meist über Formate wie JSON (JavaScript Object Notation) oder XML (Extensible Markup Language), da Aussehen und Lesbarkeit der Daten irrelevant sind und somit eine Ausgabe in HTML nicht von Nöten ist. \\
Bei der Implementierung eines Webservices bieten sich folgende zwei technologische Arten der Umsetzung an: \\ \\
\textbf{SOAP/WSDL}: Hier werden Nachrichten über das Simple Object Access Protocoll ausgetauscht (SOAP) und deren Beschreibung über die Web Services Description Language (WSDL) definiert. Anfrage- und Antwortformat der Daten ist XML (Extensible Markup Language), eine Auszeichnungssprache, welche HTML sehr ähnelt aber deutlich allgemeiner ist. XML kann als mehr als Regelwerk verstanden werden, mitdessen Hilfe Entwickler ihre eigene hierarische Beschreibung einer Datenstruktur vornehmen können. XML und HTML leiten sich bei der von der SGML (Standard Generalized Markup Language) ab, welches ihre Ähnlichkeit zusätzlich herleitet \cite{XMLHTMLU88:online}. \\

 
\textbf{REST}: (Representational State Tranfer) Hier kann jede einzelne Funktion des Webservices über eine jeweils zugeordnete URL abgerufen (Uniform Resource Locator) werden, umgangssprachlich als Webadresse bekannt. Das WWW selbst kann als REST-Webservice verstanden werden \cite{Bayer2002}. \\ 

\subsection{Webapplikationsentwicklung}\label{sec:softwareentwicklung}
\subsubsection{Vor- und Nachteile}\label{sec:vorundnachteileweb}
\subsubsection{Serverseitiger Ansatz}\label{sec:serverseitgeransatz}
%Hier über NodeJS und PHP quatschen!
\subsubsection{Clientseitiger Ansatz}\label{sec:clientseitigeransatz}
%Hier vor allem über Javascript quatschen!
\subsubsection{Hardware}\label{sec:hardware}