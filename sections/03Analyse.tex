\section{Analyse}\label{sec:analyse}
In diesem Kapitel der Arbeit werden zunächst existierende Plattformen am Markt verglichen und darauf aufbauend Anforderungen an das Projekt formuliert.
\subsection{Vergleich mit existierenden Plattformen}\label{sec:vergleichplat}
Im folgenden Abschnitt werden ausgewählte existierende Plattformen, die im Bereich 
digital gestützte Unterrichtsmethoden angesiedelt sind, beleuchtet und anschließend 
gegenübergestellt. Eine klare Trennung zwischen kommerziell und nicht-kommerziell ist schwierig bis unmöglich, da viele Plattformen im Bereich des Freemium\footnote{Freemium ist ein Geschäftsmodell, bei dem das Basisprodukt gratis angeboten wird, während das Vollprodukt und Erweiterungen kostenpflichtig sind.} Geschäftsmodells vermarktet werden. Generell gibt es sehr viele Anbieter und Plattformen und somit ist eine Beschränkung der Auswahl unabdingbar. 
% es gibt sehr sehr viele tools, eine Beschränkung im Vergleich ist unabdingbar. 
% viele tools machen dinge verschieden. Beispiel single user prinzip (mind maps)

\paragraph{SMART Learning Suite Online}
Der Anbieter SMART (Smart Technologies Corporation) ist in Deutschland vor allem für sein Angebot von interaktiven Whiteboards, welche unter dem Namen SMART Board vermarktet werden, bekannt. Ergänzend bietet SMART auch die SMART Learning Suite an, welche sowohl online als auch als lokale Installation genutzt werden kann. Positiv hervorzuheben ist, dass bei der Cloud Variante der Nutzende vorab seine Server Region festlegt. Wird hier Europa gewählt, ist anzunehmen dass europäische Richtlinien im Bezug auf Datenschutz und Speicheranforderungen berücksichtigt werden. Gespeichert werden die Daten generell auf Amazon- oder Google-Servern, wobei Smart angibt, dass in der europäischen Service Region hierbei Amazon-oder Google-Server mit Standort Deutschland genutzt werden\cite{Technologies2019}. Die SMART Learning Suite kann sowohl online als auch offline installiert werden und kostenlos getestet werden. Getestet wurde jedoch nur die online Version, da nur diese im Webbrowser läuft, welches in Hinsicht auf dieses Projekt relevant ist. \\ Das Angebot umfasst viele Funktionalitäten und unterschiedlichste Implementierungen von  interaktiven Unterrichtsmethoden, wie Quiz/Befragungen, Brainstorming, Memory, Karteikarten u.v.m. Viele Anwendungen funktionieren im Einzelanwender-Betrieb, Lehrender, Schülerinnen und Schüler nutzen das gleiche Gerät. Andere Anwendungen erfordern zusätzliche Clients, sprich Geräte wie Smartphones oder Computer, laufen also im Mehrbenutzerbetrieb. Ebenso können Lehrende Prüfungsaufgaben erstellen und diese dann Abfragen und Auswerten. Eine strikte Unterscheidung zwischen Lehrer-, Klassen-, und Studierendenansicht findet nicht statt. Ein großer Anspruch der Software ist, dass ein Lehrender ein ganzes Set an Aktivitäten für den Unterricht erstellen kann und dieses schrittweise durchlaufen wird. Es kann bspw. mit einem Test begonnen werden, anschließend erfolgt ein Folie mit einem Begriff und darauffolgend wird ein interaktive Unterrichtsmethode ausgeführt usw. 
% Sehr cool und mächtig, doof beim brainstorming, begriff steht nicht in der mitte? Sehr klein alles! Nicht lokalisiert auf Deutsch

\paragraph{ClassFlow}
Ähnlich zu SMART Learning Suite Online ist ClassFlow eine Software, welche das 
Durchführen von interaktivem Unterricht ermöglicht. Der Lehrende erstellt hierzu Sitzungen, welche ähnlich einer Präsentation durchlaufen werden. An jeder Stelle kann der Lehrende den Bildschirminhalt an die Schülerinnen und Schüler Endgeräte schicken und interaktive Unterrichtsmethoden starten, welche z.B. Umfragen, Brainstormings, Kreuzworträtsel u.v.m. sein können. Die Software läuft in der Cloud, es existiert keine Offline Variante. Für eine reine Datenspeicherung innerhalb der EU garantiert der Anbieter Promethean Limited nicht\cite{Limited2017}. Es können auch die meisten interaktiven Unterrichtsmethoden, in ClassFlow Aktivitäten genannt, im Einbenutzerbetrieb genutzt werden, d.h. die Einheit wird auf einem Computer gestartet und dort auch ausgeführt. Weitere Geräte seitens der Schülerinnen und Schüler sind dann nicht notwendig. Eine interaktive digitale Tafel ist in diesem Modus jedoch empfehlenswert. Lehrende können auch schon vorgefertigte Unterrichtseinheiten aus dem sog. Marktplatz erwerben. Es gibt kostenlose wie auch kostenpflichtige Einheiten. 

\paragraph{Google Classroom}
Die online Software Google Classroom ist eng in die Produktpalette der Firma Google inc. eingebettet. Technisch betrachtet lässt sich Google Classroom eher mit der Software Moodle vergleichen, da eher das Ziel der Organisation einer Bildungsreinrichtung bzw. derer Kurse und Klassen verfolgt wird, obgleich das erstellen von Fragen an alle Kursteilnehmer sowie von Quiz Aufgaben möglich ist. Beim Quiz wird die hauseigene online Software Google Formulare verwendet. Bildungseinrichtungen müssen sich zunächst als solche registrieren, bevor eine Nutzung erlaubt wird. Man kann Google Classroom allerdings auch privat mit einem Google Konto nutzen, wenn explizt angegeben wird, dass die Software nicht in einer Bildungseinrichtung genutzt wird. Bildungsreinrichtungen müssen ein G Suite for Education Konto eröffnen, welches die Verwendung und Verwaltung weiterer Google Software mit sich bringt, so z.B. Google Kalender, G-Mail und weitere. Eine analoges Softwareangebot besteht auch für Unternehme namens G Suite. Der Google eigene Cloudspeicherdienst Google Drive ist angebunden und somit werden z.B. erstellte Quizze dort abgespeichert.  Beim Speichern der Daten kann von nicht EU-zentralen Google eigenen Cloud-Servern ausgegangen werden. 

\paragraph{Sonstige}
Neben den o.g. größeren Anbietern existieren viele kleinere Online Angebote, welche sich meist auf die Bereitstellung einer Dienstleistung bzw. Ausführung einer interaktiven Unterrichtsmethode beschränke, hierbei jedoch oftmals auch interessante Ansätze zu finden sind. Gerade wenn Dozierende unkompliziert eine bestimmte interaktive Unterrichtsmehthode im Unterricht einsetzen möchte, bietet es sich an auf einen kleineren Anbieter zurückzugreifen. Zu nennen wäre hier z.B. die Software \textbf{Plickers}, welche das Prinzip bring-your-own-device etwas abändert. Die Schülerinnen und Schüler benötigen hier lediglich Papier auf dem spezielle QR-Codes abgebildet sind. Bei Fragestellungen wird das Papier nach oben gehalten und je nachdem welche Seite des Papiers (und somit auch des QR-Codes) nach oben zeigt, wird entschieden ob für Antwort A, B, C oder D plädiert wird. Eine Kamera vom Smartphone oder Tablet des Dozierenden erkennt dies und kann somit die Daten auswerten. Ähnlich verfährt die Anwendung \textbf{Poll Everywhere}, hier ist allerdings ein Endgerät pro Schülerin bzw. Schüler notwending. Wer nur teilnimmt muss jedoch keine Applikation installieren, hier reicht ein spezieller Link der in einem Webbrowser aufgerufen wird.  
% Freemium modell auch, hier geben die Leute via app die antwort
% geteiltes Client prinzip (Presenter and teacher view)
% Gezielt für Umfragen, nicht quiz
% wohl kostenlos
\subsubsection{Gegenüberstellung}\label{sec:gegenstellung}
Nach der Präsentation der ausgewählten Angebote in Abschnitt \ref{sec:vergleichplat} werden diese in folgenden Hauptkriterien miteinander verglichen:
\begin{enumerate}
	\item \textbf{Serverstandort}: Werden die Server des Anbieters innerhalb der Europäischen Union betrieben, sodass datenschutzrechtliche Regelungen dieser Anwendungen finden?
	\item \textbf{Online Nutzung}: Ist die Nutzung über das Internet möglich? 
	\item \textbf{Offline Nutzung}: Ist die Nutzung offline über das Intranet möglich? 
	\item \textbf{Betriebsart E/M}: Ist die Nutzung im Einzelbenutzerbetrieb und/oder im Mehrbenutzerbetrieb möglich? Ersteres bedeute, dass eventuell mehrere Nutzer die Software ggf. an einem Computer verwenden können, eine Interaktion über mehrere angebundene Clients ist nicht möglich. Im Mehrbenutzerbetrieb können sich können sich Nutzende über Clients mit der Software verbinden und gemeinsam interaktiv werden.
	\item \textbf{Betriebsmodus (Single/Set)}: Können mehrere Unterrichtsmethoden als Set angelegt werden, welches später sukzessiv durchlaufen wird oder können nur einzeln angelegte Unterrichtsmethoden nach und nach manuell gestartet werden? (Single)
	\item \textbf{Clients}: Gibt es unterschiedliche  Arten von Clients für Lehrende, Teilnehmende und spezielle, die nur zur Anzeige gedacht sind?
	\item \textbf{Registrierung}: Ist für die Nutzung eine Registrierung für Lehrende notwendig? Ebenso für Schülerinnen und Schüler?
	\item \textbf{Unterstützte Unterrichtsmethoden}: Welche lassen sich nutzen? Bei mehr als zwei wird hier die reine Zahl genannt  
\end{enumerate}

% \usepackage{booktabs}
\begin{table}[h!]
	\caption{Tabellarischer Vergleich existierender Plattformen}
	\label{tab:vergplatt}
	\resizebox{\textwidth}{!}{%
	\begin{tabular}{@{}|l|c|c|c|c|c|@{}}
		\toprule
		\textbf{Produkt}    & \textbf{SMART LSO}          & \textbf{ClassFlow}           & \textbf{Google Classroom}    & \textbf{Plickers}            & \textbf{Poll Everywhere}          \\ \midrule
		Serverstandort EU   & \cellcolor[HTML]{32CB00}Ja* & \cellcolor[HTML]{FE0000}Nein & \cellcolor[HTML]{FE0000}Nein & \cellcolor[HTML]{FE0000}Nein & \cellcolor[HTML]{FFCC67}Unbekannt \\ \midrule
		Online Nutztung     & \cellcolor[HTML]{32CB00}Ja  & \cellcolor[HTML]{32CB00}Ja   & \cellcolor[HTML]{32CB00}Ja   & \cellcolor[HTML]{32CB00}Ja   & \cellcolor[HTML]{32CB00}Ja        \\ \midrule
		Offline Nutzung     & \cellcolor[HTML]{32CB00}Ja  & \cellcolor[HTML]{FE0000}Nein & \cellcolor[HTML]{FE0000}Nein & \cellcolor[HTML]{FE0000}Nein & \cellcolor[HTML]{FE0000}Nein      \\ \midrule
		Betriebsmodus       & Single/Set                  & Single/Set                   & Single                       & Single                       & Sinlgle                           \\ \midrule
		Clients             & 2 (Lehrer/Schüler)          & 2 (Lehrer/Schüler)           & 2 (Lehrer/Schüler)           & 2 (Lehrer/Schüler)**         & 2 (Lehrer/Schüler)                \\ \midrule
		Registrierung       & Ja/Nein                     & Ja/Ja                        & Ja/Ja                        & Ja/Nein                      & Ja/Nein                           \\ \midrule
		Unterrichtsmethoden & 12                          & 10                           & Frage/Quiz                   & Quiz                         & Umfragen                          \\ \bottomrule
	\end{tabular}
}
\footnotesize {
	* Option ist innerhalb der Software wählbar. \\
	** Der 'Client' für die Schülerinnen und Schüler stellt in diesem Fall Papier mit aufgedruckten QR Codes da.
}

\end{table}

Aus der Tabelle lässt sich erschließen, dass SMART Learning Suite Online und ClassFlow
HIER GEHTS WEITER

\subsection{Systembeschreibung}\label{sec:sysbeschreib}
Aus Gründen der Lesbarkeit wurde im folgenden Abschnitt die männliche Form gewählt, nichtsdestoweniger beziehen sich die Angaben auf Angehörige beider Geschlechter. 

Die Software soll als Web Server-Applikation implementiert sein. Über drei verschiedene Client Lösungen kann mit dem Server interagiert werden. \\ \\

1.) Über ein Backend Zugang können Administratoren und Dozierende den Server verwalten sowie erstmalig initialisieren. Neue Dozierende können einen Benutzeraccount anlegen, welcher von Administratoren freigeschaltet werden muss. Alternativ können Administratoren neue Benutzerkonten anlegen. Dozierende ist es möglich Lehreinheiten zu erstellen, diese zu starten sowie zu beenden. Während einer aktiven Lehreinheit, ist es Dozierenden möglich, diese zu leiten. (Fortschritt, Verbundene Schüler/Studenten verwalten, Speichern, je nach Typ der Lehreinheit.) Eine Lehreinheit ist eine interaktive Unterrichtsmethodik softwareseitig umgesetzt. Als erste Umsetzung erfolgt in diesem Projekt die Implementierung eines Brainstorming und Quiz. Der parallele Betrieb von mehreren, unabhängig am System authentifizierten Lehrenden, welche Lehreinheiten starten und zu denen sich Schülerinnen und Schülern einschreiben, soll möglich sein \\ \\

2.) Ein Präsenter Client zeigt die zur Laufzeit einer aktiven Lehreinheit relevanten Informationen an. Dieser ist zur Anzeige auf einem Großflächenanzeigegerät ausgelegt (Projektor, Fernseher, Smart Board). Es soll problemlos möglich sein, mehrere Präsente Clients an anzukoppeln. \\ \\
3.) Schüler/Studenten geben einen frei wählbaren Namen an. Ein Benutzerregistrierung ist nicht notwendig. Sie können anschließend aktiven Lehreinheiten beitreten und nach dem Start an diesen partizipieren. \\ \\

Eine detaillierte Aufführung der Anforderungen und Eigenschaften dieses Projekts erfolgt in den nachfolgenden Abschnitten. 

\subsection{Zielgruppe}\label{sec:zielgruppe}
Das Software-Projekt soll sich in erster Linie an Bildungseinrichtungen jeglicher Art und deren Dozierenden richten, welche eine lokal ausgeführte Softwarelösung für das Durchführen von interaktiven Unterrichtsmethoden bevorzugen. Darüber hinaus auch an jegliche, die digital gestützte interaktive Lern- und kompetitive Kleinstspiele nutzen möchten. Dabei ist eine flexible Skalierbarkeit des Servers gegeben (siehe Abschnitt \ref{sec:anforderung}). Desweiteren ist das Software-Projekt attraktiv für die Open-Source Community, welche das Projekt weiter ausbauen kann sowie neue Typen von Lehreinheiten (interaktive Unterrichtsmethode softwareseitig umgesetzt / 'Spiel'-Art) hinzufügen kann (weiterführend REF UNBEKANNT NOCH!.) 
\subsection{Abgrenzung}\label{sec:abgrenz}
Der Prototyp des Softwareprojekts (interne Bezeichnung Node ICT \footnote{Node ICT steht für Node.js interactive course teaching. Das Node.js Server Framework bildet den Grundstein des Softwareprojekts}) soll das Anlegen und Ausführen von Lehreinheiten ermöglichen. Der Prototyp wird auf dem lokalen Host getestet (Server und Client auf derselbe Maschine ausgeführt) sowie im Intranet (LAN, Server und Clients auf unterschiedlichen Maschinen ausgeführt). Eine verschlüsselte Kommunikation zwischen Server und Client ist erwünscht, wird jedoch nicht im Prototyp implementiert. Eine Nutzung über öffentlicher IP-Adresse oder Domain im Internet ist prinzipiell möglich, wird jedoch nicht getestet. Ebenso wird bei dem Prototyp vermindert Augenmerk auf das Design der Client Anwendungen gelegt, worunter die Usability jedoch nicht leiden soll. Das Design wird leicht anpassbar sein. Wie in Abschnitt \ref{sec:sysbeschreib} erwähnt, wird sich auf das Implementieren von zwei interaktiven Unterrichtsmethoden beschränkt, der Prototyp wird aber das Umsetzen und Hinzufügen weitere Unterrichtsmethoden ermöglichen, da prinzipielle nur die Logik der neuen Unterrichtsmethode geschrieben werden muss. Eine Wiederverwendbarkeit von grundlegender Funktionalität (z.B. Anbindung an Datenbank, Management verbundener Clients etc.) wird bereitgestellt.   
\subsection{Systemanforderungen}\label{sec:anforderung}
Aufbauend auf das Analysekapitel dieser Arbeit (Abschnitt \ref{sec:analyse} ff.) sowie dem Abschnitten \ref{sec:digianschulen} und \ref{sec:datenschutz} lassen sich funktionale und nicht-funktionale Anforderungen an das System ausformulieren, welche in den folgenden zwei Abschnitten gelistet werden.

\subsubsection{Funktional}\label{sec:anffunc}
//TODO tabelle
\subsubsection{Nicht Funktional}\label{sec:nichtfunc}
//TODO tabelle
\subsection{Technische Anforderungen}\label{sec:techanford}
In diesem Abschnitt werden die technischen Anforderungen an die Hardware erläutert, welche einen reibungslosen Betriebsablauf gewährleisten sollen. Es wird hierbei zwischen Server und Client Anforderungen unterschieden obgleich Server und Client auch auf der selben Maschine ausgeführt werden können. 
\subsubsection{Server}\label{sec:servertech}
Die Server Software soll so umgesetzt werden, dass sie auch auf leistungsschwächerer Hardware problemlos mehrere Benutzer gleichzeitig ohne signifikante Performanceeinbuße bedienen kann. Die Hardwarespezifikationen eines Raspberry Pi (Version 3B) Einplantinencomputer (Querverweis zu Abschnitt \ref{sec:einfuhrung}) soll hierbei als Mindestanforderung definiert sein. Durch die Nutzung des Serverwebframeworks 'Node.js' als Grundgerüst der Software, ist ein plattformunabhängiger Betrieb gewährleistet. Der Server soll rein im Intranet lauffähig sein und keine online Abhängigkeiten besitzen. Daher soll technisch keine (Breitband)Internetanbindung für den Betrieb notwendig sein. Es ist ebenso soll es möglich sein den Server 'headless', d.h. ohne angeschlossene Peripheriegeräte wie Tastatur, Maus und Bildschirm zu betreiben. Die Initialisierung der Software kann hierbei durch ein Startskript oder beispielsweise über einen SSH\footnote{Mittels SSH (Secure Shell) kann eine verschlüsselte Verbindung zur Kommandozeile (Shell) auf einem Server hergestellt werden} Zugang erfolgen.
\subsubsection{Client}\label{sec:clienttech}
Der im Rahmen dieses Projekts zu entwickelnde Client Software kann wie in Abschnitt \ref{sec:sysbeschreib} erläutert, in drei Parts eingeteilt werden. Die Verwaltung im Backendbereichs der Server Software soll eine besonders niedrige Hardwareanforderung aufweisen, da hier gänzlich auf den Einsatz von JavaScript verzichtet werden wird. Dies soll eine Server-Verwaltung auch bei deaktiviertem JavaScript gewährleisten. Die restlichen Software Module sollen in jedem modernen Webbrowser auf jedem Endgerät lauffähig sein. Der Einsatz von JavaScript ist hier unverzichtbar. Eine Kompatibilitätsabdeckung von 95\% zur ECMAScript 6 (ECMAScript 2015) Spezifikation sollte vom verwendeten Webbrowser gegeben sein. Diese Anforderung erfüllen jedoch alle modernen Webbrowser (Stand 2019)\cite{ECMAScri7:online}. 
