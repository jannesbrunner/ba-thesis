\part*{Anhang}
\subsection*{Abschnitt: Funktionale Anforderungen}

\begin{table}[h!]
	\caption{Funktionale Anforderungen an die Projektsoftware}
	\label{tab:funkanf}
	\resizebox{\textwidth}{!}{%
		% Please add the following required packages to your document preamble:
		% \usepackage{booktabs}
		% \usepackage[table,xcdraw]{xcolor}
		% If you use beamer only pass "xcolor=table" option, i.e. \documentclass[xcolor=table]{beamer}
		\begin{tabular}{@{}cllc@{}}
			
			\toprule
			\rowcolor[HTML]{FFFFFF} 
			\textbf{ID} & \multicolumn{1}{c}{\cellcolor[HTML]{FFFFFF}\textbf{Name}} & \multicolumn{1}{c}{\cellcolor[HTML]{FFFFFF}\textbf{Beschreibung}}                                                                                                                                                                                                                                                                 & \textbf{Status} \\ \midrule
			\rowcolor[HTML]{FFFFFF} 
			F01         & Datenbankanbindung                                        & Es soll erfolgreich eine Anbindung an die Datenbank erfolgen                                                                                                                                                                                                                                                                      & OK              \\
			\rowcolor[HTML]{EFEFEF} 
			F02         & Datenbank Generierung                                     & \begin{tabular}[c]{@{}l@{}}Alle nötigen Modelle und Tabellen soll in der Datenbank\\ abgebildet werden und bei bedarf gänzlich neu generiert werden\end{tabular}                                                                                                                                                                  & OK              \\
			\rowcolor[HTML]{FFFFFF} 
			F03         & GET Requests                                              & \begin{tabular}[c]{@{}l@{}}Der Server soll in der Lage sein GET Requests entgegenzunehmen\\ und zu beantworten. Im Fehlerfall soll auch eine Antwort erfolgen.\end{tabular}                                                                                                                                                       & OK              \\
			\rowcolor[HTML]{EFEFEF} 
			F04         & POST Requests                                             & \begin{tabular}[c]{@{}l@{}}Der Server soll in der Lage sein POST Requests entgegenzunehmen\\ und zu beantworten. Im Fehlerfall soll auch eine Antwort erfolgen.\end{tabular}                                                                                                                                                      & OK              \\
			\rowcolor[HTML]{FFFFFF} 
			F05         & Login System                                              & \begin{tabular}[c]{@{}l@{}}Nutzende der Software in der Rolle als Lehrkraft oder Administrator\\ sollen in der Lage sein sich bei dem System mittels Authentifizierung\\ an- und abzumelden mittels HTTP-Session und Session-Cookies\end{tabular}                                                                                 & OK              \\
			\rowcolor[HTML]{EFEFEF} 
			F06         & MVC - Pattern                                             & \begin{tabular}[c]{@{}l@{}}Das Model-View-Controller Muster soll beim Datenfluss im Backend/\\ Lehrerbereich des Servers reflektiert und zum Einsatz kommen\end{tabular}                                                                                                                                                          & OK              \\
			\rowcolor[HTML]{FFFFFF} 
			F07         & WebSockets                                                & \begin{tabular}[c]{@{}l@{}}Bei der Ausführung von Lehreineheiten und den damit verbundenen\\ Unterrichtsmethoden soll die Kommunikation zwischen \\ Server und Web-Client über das WebSocket Protokoll erfolgen\end{tabular}                                                                                                      & OK              \\
			\rowcolor[HTML]{EFEFEF} 
			F08         & Drei Client Implementierung                               & \begin{tabular}[c]{@{}l@{}}Um die Software größtmöglichst flexibel einsetzen zu können, sollen\\ drei verschieden optimierte Web-Clients implementiert werden, genauer\\ einen für die ausführende Lehrkraft, einen für Studierende, eine für \\ Anzeigemedien, auf größere Darstellung optimiert im Unterrichtsraum\end{tabular} & OK              \\
			\rowcolor[HTML]{FFFFFF} 
			F09         & Administration                                            & \begin{tabular}[c]{@{}l@{}}Authentifizierte Nutzer wie Dozierende und Administratoren sollen \\ die Software verwalten können, dies umfasst u.A. das Anlegen und \\ Freischalten von Nutzenden und das Setzen der Werkseinstellungen\end{tabular}                                                                                 & OK              \\
			\rowcolor[HTML]{EFEFEF} 
			F10         & Absicherung                                               & \begin{tabular}[c]{@{}l@{}}Alle Routen die höhere Privilegien verlangen, sollen nur authentifizierten\\ Nutzenden zugänglich gemacht werden, dies betrifft vor allem F09 und\\ Besitzer abhängig erstellte Ressourcen\end{tabular}                                                                                                & OK              \\
			\rowcolor[HTML]{FFFFFF} 
			F11         & Ausführung von Lehreinheiten                              & \begin{tabular}[c]{@{}l@{}}Lehrkräften bzw. Dozierende soll es möglich sein zwei Typen von\\ interaktiven Unterrichtsmethoden mit Studierenden durchzuführen, \\ dies umfasst zunächst die Typen Brainstorming und Quiz\end{tabular}                                                                                              & OK              \\
			\rowcolor[HTML]{EFEFEF} 
			F12         & Einfacher Studierendenzugang                              & \begin{tabular}[c]{@{}l@{}}Studierende bzw. Schülerinnen und Schüler sollen über einen QR Code\\ vereinfacht Zugriff auf den Server erhalten, vorausgesetzt sie sind mit \\ dem gleichem Netzwerk wie der Server verbunden\end{tabular}                                                                                           & OK              \\
			\rowcolor[HTML]{FFFFFF} 
			F13         & Wifi/WLAN Hotspot                                         & \begin{tabular}[c]{@{}l@{}}Der Server soll ein eigenes WLAN Netzwerk bereitstellen können in \\ dem der Host-Computer des Servers als WLAN Access Point fungiert, \\ und verbundenen Clients automatisch eine IP-Adresse zuweist (DHCP)\end{tabular}                                                                              & N.I.*        \\ \bottomrule
		\end{tabular}
	}
	\footnotesize * nicht implementiert
\end{table}