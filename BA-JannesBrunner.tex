\documentclass[12pt,a4paper,bibtotoc,pointlessnumbers]{scrartcl}
\usepackage[ngerman]{babel}
%\usepackage[german]{polyglossia}
%\selectlanguage{german}
% ##########################################
% PDFLaTeX oder nicht:
\newif \ifPDF                                   
\ifx \pdfoutput \undifined \PDFfalse 
\else \ifnum \pdfoutput >0 \PDFtrue
     \else \PDFfalse         
     \fi     
\fi 
% ##########################################
\usepackage{lmodern} %ersetzt Standarsschrift --> für schönere Textdarstellung im PDF
\usepackage[onehalfspacing]{setspace} %setzt das Dokument in 1,5-fachem Zeilenabstand
\setlength\parindent{0pt}% Setzt den Einzug nach einem Absatz fest (hier auf Null)
\usepackage{microtype}

%\usepackage{csquotes}
%\usepackage[style=acm]{biblatex}
%\addbibresource{Literaur.bib}
\usepackage{bibgerm}
%\usepackage{natbib}

\usepackage[
backend=biber,
style=numeric,
%-verb
sorting=none
]{biblatex}
\addbibresource{BA-JannesBrunner.bib}
%\addbibresource{Bildquellen.bib}
%%\bibliography{Masterarbeit.bib}
%\usepackage{cite}


\usepackage[T1]{fontenc} %für europäischen ASCII-Code (mit Umlauten etc)
\usepackage[utf8]{inputenc} %Legt den Zeichencode fest

\usepackage[babel, german=quotes]{csquotes}

\usepackage[dvips]{graphicx} %damit Bilder eingebunden werden Können
\usepackage{fancyhdr} %Nachfolger zu "fancyheadings" --> legt Layout der Seite fest + hält zusätzliche Befehle und Optionen bereit

%\usepackage[headsepline]{scrlayer-scrpage}
%\automark[section]{subsection}

\usepackage{float} %Objekte, die nicht auf zwei Seiten aufgeteilt werden sollen (z.B. Bilder, Tabellen)
\usepackage{floatflt} %wird für die Umgebung floatfigure benötigt --> Bild kann von text umflossen werden
\usepackage[section]{placeins} %section verhindert, dass Bilder aus section hinausgleiten

\usepackage{amsmath} %für Matheumgebung
\usepackage{amssymb} %Setzt Befehle für Symbole in Symbole um 
\usepackage{amsfonts} %Zusätzliche Schriften und Symbole
\usepackage{exscale} %stellt Befehle für Änderung der Schrift bereit
\usepackage{textcomp}
\usepackage{easymat} % weitere Befehle für Matrizen
\usepackage{enumitem} % Ändern von Listenumgebungen
%\usepackage{paralist}

\usepackage{pdfpages}%erlaubt das einbinden von anderen PDF-Dokumenten (auch einzelnen Seiten) in das Latex-Dokument

\usepackage{etex} %erhöht die Anzahl der Pakete, die geladen werden können

\usepackage[font=small,labelfont=bf,hang]{caption} %für Bild- und Tabellenunterschriften
\addto\captionsngerman
{
	\renewcommand{\figurename}{\small{\textbf{Abb.}}}
	\captionsetup{figurewithin = section}
}
\addto\captionsngerman{
	\renewcommand{\tablename}{\small{Tab.}}}
\captionsetup{tablewithin = section}
\captionsetup{font=small, labelfont=bf}

%\usepackage{subfig}
\usepackage{subcaption}
\usepackage{longtable}% erlaubt Tabellen über mehrere Seiten
\usepackage{tabularx} %Tabellen können größer gemacht werden und und die Spalten sind flexibler
\usepackage{booktabs} %definiert Befehle für Tabellen
%\usepackage{booktabs-de}
\usepackage{ltxtable} %vereinigt longtable und tabularx
\usepackage{colortbl}
\usepackage{multirow}
%\usepackage{multicolumn}
\usepackage{rotating}
%\usepackage{tablefootnote}

%\usepackage{tikz}

\usepackage{array} %erweiterte Möglichkeiten für Tabellen und Spalten
%\usepackage{subfigure}
\usepackage{url} %lange Zeichenfolgen können trotz fehlender Leerzeichen getrennt werden
%\usepackage{paralist}
% ##########################################

% ##########################################
% Seitendesign:
\usepackage{units}
\usepackage[headheight=50pt, a4paper, bottom=25mm, footskip=6mm]{geometry} %wird für Layout des Dokuments benötigt
\geometry{verbose}
% ##########################################
%\usepackage{cite} % fasst Zitate zusammen (funktioniert nur ohne hyperref)

\ifPDF
\usepackage{xcolor}   %ermöglicht das Ändern der Farbe von Schrift, Seitenhintergrund, Boxen etc
\usepackage[
  pdftex,
  colorlinks,
  citecolor = {green},
  linkcolor = {black},
  urlcolor  = {gray},                
  bookmarks         = true,
  bookmarksopen     = true, % Bookmarks anzeigen...
  bookmarksnumbered = true, % ...und numerieren
  pdftitle          = {Masterarbeit},
  pdfauthor         = {Nanin Brunner}
]{hyperref} %ermöglicht Links im PDF-Dokument (z.B. ancklicken eines Kapitel im Inhaltsverzeichnis oder Verlinkung mit externem Link)


\else
\usepackage{wrapfig}
%\usepackage{floatflt} %ermöglicht Gleitobjekte (Bilder, Tabellen) von Text umfließen zu lassen und Bilder links/recht je nach Seitenzahl auszurichten
\fi%-----------------------------------------------------------------------------
% Definition für einen neuen Befehl für deutsche Anführungszeichen
\newcommand{\gqq}[1]{\glqq{}#1\grqq{}}
\newcommand{\gq}[1]{\glq{}#1\grq{}}


\begin{document}
	\begin{titlepage}
		\begin{center}
%				\begin{figure}[h]
%					\begin{minipage}{.4\textwidth}
%						\centering
%						\includegraphics[width=0.5\textwidth]{Bilder/Logo_Wildau}
%					\end{minipage}
%					\begin{minipage}{.2\textwidth}
%						\hspace{\textwidth}
%					\end{minipage}
%					\begin{minipage}{.4\textwidth}
%						\centering
%						\includegraphics[width=0.4\textwidth]{Bilder/bam_logo_135}
%					\end{minipage}
%				\end{figure}			
%				\vspace{0.4cm}
			
%			{\large \bf Bundesanstalt für Materialforschung und -prüfung }\\[4mm]
%			{\large\bf TH Wildau}
			%\vspace{0.5cm} \hrule \vspace{1.3cm}
			
			 
			\begin{doublespace}
			{\huge Titel der Arbeit}\\
			\end{doublespace}
			\vspace{2cm}
			
			{\huge \bf Bachelorarbeit}\\
			\vspace{0.4cm}
			
			{\large zur Erlangung des akademischen Grades\\ Master\\ (M.-Eng.)\\an der Technischen Hochschule Wildau}\\
			
			\vspace{1cm}
			
			{\Large \bf Technische Hochschule Wildau}\\
			{\large \bf Fachbereich Ingenieur- und Naturwissenschaften\\
			Studiengang Photonik}\\
		
			\vspace{1.5cm}
			
			{\doublespacing Eingereicht von\\
			{\Large \bf Marie Nanine Brunner}\\ geb. 13.08.1983}
			
%			\vspace{1cm}
		
			\large
			\begin{table}[b]
				\begin{center}
					\begin{tabular}{ll}
					Eingereicht am:& 10.03.2019\\
					Betreuender Hochschuldozent:& Prof. Dr. sc. nat. Klaus-Peter Möllmann\\
					Themenstellende Einrichtung:& Bundesanstalt für Materialforschung und -prüfung\\
					Zweitgutachter: & Dipl.-Physiker Gerd Wilsch (BAM)\\
					\end{tabular}
				\end{center}
			\end{table} 
		\end{center}
		\vspace{0.6cm}
	%	{\bf Abstract}: Das Abstract
\end{titlepage}

\newpage
%\clearpairofpagestyles




%\clearpairofpagestyles

\pagenumbering{roman}
\setcounter{page}{10}


%\thispagestyle{empty}
\cfoot{\pagemark}
\tableofcontents 
\newpage

\pagenumbering{arabic}




%\pagestyle{scrheadings}
%\ihead{\leftmark}
%%\automark{section}
%\chead{\rightmark}
%%\automark{subsection}
%\ohead{\thepage}


\pagestyle{fancy}
\fancyhf{}
%\setlength{\headheight}{15pt}
\renewcommand{\headrulewidth}{0.4pt} %definiert die Dicke der Linie unter der Kopfzeile
%\renewcommand{\footrulewidth}{0pt} %definiert die Dicke der Linie über der Fußzeile
\renewcommand{\sectionmark}[1]{%
	\markboth{\thesection.\ #1}{}} %sorgt dafüpr, dass die Kapitel und Unterkapitel in der Kopfzeile stehen


%\addtolength{\headwidth}{\marginparwidth}
%\setlength{\fancyhead}{0.4\headrulewidth}
\fancyhead[L]{\textsl{\small 
%		\leftmark \hspace{0.8cm}
		\rightmark}}
%\fancyhead[C]{\parbox{0.5\textwidth}{\textsl{\small \rightmark}}}
\fancyhead[R]{\textsl{\small \thepage}}
%\fancyhead[L]{\parbox{0.3\textwidth}{\textsl{\small \leftmark}}}
%\fancyhead[C]{\parbox{0.6\textwidth}{\textsl{\small \rightmark}}}
%\fancyhead[R]{\parbox{0.05\textwidth}{\textsl{\small \thepage}}}
%\addtolength{\headwidth}{\marginparsep}
%\fancyfoot[C]{\thepage}


\input{sections/00Begriffe}
\newpage

\section{Test}


Hallo das hier ist ein Test \cite{schoning_theoretische_1992}



%\fancyhead[L]{\textsl{\small \leftmark}}
%\input{sections/07Zusammenfassung.tex}%\newpage
%\input{sections/08Ausblick.tex}\newpage

%\addcontentsline{toc}{section}{Literaturverzeichnis}
%\printbibliography[nottype=online, title={Literaturverzeichnis}]
%\printbibliography[type=online,title={Online - Bildquellen}]\newpage

\printbibliography[title={Literaturverzeichnis}]
\newpage


\addcontentsline{toc}{section}{Abbildungsverzeichnis}
\listoffigures\newpage
\addcontentsline{toc}{section}{Tabellenverzeichnis}
\listoftables

%\clearpage%\vspace*{-3cm}
%\newpage

\addcontentsline{toc}{part}{Anhang}

\fancyhead[L]{\textsl{\small \leftmark \hspace{0.8cm}\rightmark}}

%\appendix % Für Anhänge
%\input{anhang/Zemente}
%\input{anhang/Nachmessung_Zemente}



\end{document}