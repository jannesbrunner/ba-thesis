\documentclass[12pt,a4paper,bibtotoc,pointlessnumbers]{scrartcl}
\usepackage[ngerman]{babel}
%\usepackage[german]{polyglossia}
%\selectlanguage{german}
% ##########################################
% PDFLaTeX oder nicht:
\newif \ifPDF                                   
\ifx \pdfoutput \undifined \PDFfalse 
\else \ifnum \pdfoutput >0 \PDFtrue
     \else \PDFfalse         
     \fi     
\fi 
% ##########################################
\usepackage{lmodern} %ersetzt Standarsschrift --> für schönere Textdarstellung im PDF
\usepackage[onehalfspacing]{setspace} %setzt das Dokument in 1,5-fachem Zeilenabstand
\setlength\parindent{0pt}% Setzt den Einzug nach einem Absatz fest (hier auf Null)
\usepackage{microtype}

%\usepackage{csquotes}
%\usepackage[style=acm]{biblatex}
%\addbibresource{Literaur.bib}
\usepackage{bibgerm}
%\usepackage{natbib}

\usepackage[
backend=biber,
style=numeric,
%-verb
sorting=none
]{biblatex}

\addbibresource{BA-JannesBrunner.bib}
%\addbibresource{Bildquellen.bib}
%%\bibliography{Masterarbeit.bib}
%\usepackage{cite}


\usepackage[T1]{fontenc} %für europäischen ASCII-Code (mit Umlauten etc)
\usepackage[utf8]{inputenc} %Legt den Zeichencode fest

\usepackage[babel, german=quotes]{csquotes}

\usepackage[dvips]{graphicx} %damit Bilder eingebunden werden Können
\usepackage{fancyhdr} %Nachfolger zu "fancyheadings" --> legt Layout der Seite fest + hält zusätzliche Befehle und Optionen bereit

%\usepackage[headsepline]{scrlayer-scrpage}
%\automark[section]{subsection}

\usepackage{float} %Objekte, die nicht auf zwei Seiten aufgeteilt werden sollen (z.B. Bilder, Tabellen)
\usepackage{floatflt} %wird für die Umgebung floatfigure benötigt --> Bild kann von text umflossen werden
\usepackage[section]{placeins} %section verhindert, dass Bilder aus section hinausgleiten

\usepackage{amsmath} %für Matheumgebung
\usepackage{amssymb} %Setzt Befehle für Symbole in Symbole um 
\usepackage{amsfonts} %Zusätzliche Schriften und Symbole
\usepackage{exscale} %stellt Befehle für Änderung der Schrift bereit
\usepackage{textcomp}
\usepackage{easymat} % weitere Befehle für Matrizen
\usepackage{enumitem} % Ändern von Listenumgebungen
%\usepackage{paralist}

\usepackage{pdfpages}%erlaubt das einbinden von anderen PDF-Dokumenten (auch einzelnen Seiten) in das Latex-Dokument

\usepackage{etex} %erhöht die Anzahl der Pakete, die geladen werden können

\usepackage[font=small,labelfont=bf,hang]{caption} %für Bild- und Tabellenunterschriften
\addto\captionsngerman
{
	\renewcommand{\figurename}{\small{\textbf{Abb.}}}
	\captionsetup{figurewithin = section}
}
\addto\captionsngerman{
	\renewcommand{\tablename}{\small{Tab.}}}
\captionsetup{tablewithin = section}
\captionsetup{font=small, labelfont=bf}

%\usepackage{subfig}
\usepackage{subcaption}
\usepackage{longtable}% erlaubt Tabellen über mehrere Seiten
\usepackage{tabularx} %Tabellen können größer gemacht werden und und die Spalten sind flexibler
\usepackage{booktabs} %definiert Befehle für Tabellen
%\usepackage{booktabs-de}
\usepackage{ltxtable} %vereinigt longtable und tabularx
\usepackage{colortbl}
\usepackage{multirow}
%\usepackage{multicolumn}
\usepackage{rotating}
%\usepackage{tablefootnote}

%\usepackage{tikz}

\usepackage{array} %erweiterte Möglichkeiten für Tabellen und Spalten
%\usepackage{subfigure}
\usepackage{url} %lange Zeichenfolgen können trotz fehlender Leerzeichen getrennt werden
%\usepackage{paralist}
% ##########################################

% ##########################################
% Seitendesign:
\usepackage{units}
\usepackage[headheight=50pt, a4paper, bottom=25mm, footskip=6mm]{geometry} %wird für Layout des Dokuments benötigt
\geometry{verbose}
% ##########################################
%\usepackage{cite} % fasst Zitate zusammen (funktioniert nur ohne hyperref)

\ifPDF
\usepackage{xcolor}   %ermöglicht das Ändern der Farbe von Schrift, Seitenhintergrund, Boxen etc
\usepackage[
  pdftex,
  colorlinks,
  citecolor = {green},
  linkcolor = {black},
  urlcolor  = {gray},                
  bookmarks         = true,
  bookmarksopen     = true, % Bookmarks anzeigen...
  bookmarksnumbered = true, % ...und numerieren
  pdftitle          = {BA},
  pdfauthor         = {Jannes Brunner}
]{hyperref} %ermöglicht Links im PDF-Dokument (z.B. ancklicken eines Kapitel im Inhaltsverzeichnis oder Verlinkung mit externem Link)


\else
\usepackage{wrapfig}
%\usepackage{floatflt} %ermöglicht Gleitobjekte (Bilder, Tabellen) von Text umfließen zu lassen und Bilder links/recht je nach Seitenzahl auszurichten
\fi%-----------------------------------------------------------------------------
% Definition für einen neuen Befehl für deutsche Anführungszeichen
\newcommand{\gqq}[1]{\glqq{}#1\grqq{}}
\newcommand{\gq}[1]{\glq{}#1\grq{}}


\begin{document}
	\begin{titlepage}
		\begin{center}
%				\begin{figure}[h]
%					\begin{minipage}{.4\textwidth}
%						\centering
%						\includegraphics[width=0.5\textwidth]{Bilder/Logo_Wildau}
%					\end{minipage}
%					\begin{minipage}{.2\textwidth}
%						\hspace{\textwidth}
%					\end{minipage}
%					\begin{minipage}{.4\textwidth}
%						\centering
%						\includegraphics[width=0.4\textwidth]{Bilder/bam_logo_135}
%					\end{minipage}
%				\end{figure}			
%				\vspace{0.4cm}
			
%			{\large \bf Bundesanstalt für Materialforschung und -prüfung }\\[4mm]
%			{\large\bf TH Wildau}
			%\vspace{0.5cm} \hrule \vspace{1.3cm}
			
			 
			\begin{doublespace}
			{\huge Entwicklung einer webbasierten Client-Server Anwendung zur Unterstützung von interaktiven Unterrichtsmethoden}\\
			\end{doublespace}
			\vspace{2cm}
			
			{\huge \bf Bachelorarbeit}\\
			\vspace{0.4cm}
			
			{\large zur Erlangung des akademischen Grades\\ Bachelor\\ (B.-Sc.)\\an der HTW Berlin}\\
			
			\vspace{1cm}
			
			{\Large \bf Hochschule für Technik und Wirtschaft Berlin}\\
			{\large \bf Fachbereich Informatik, Kommunikation und Wirtschaft\\
			Studiengang internationale Medieninformatik}\\
		
			\vspace{1.5cm}
			
			{\doublespacing Eingereicht von\\
			{\Large \bf Jannes Julian Brunner}\\ geb. 21.06.1991}
			
%			\vspace{1cm}
		
			\large
			\begin{table}[b]
				\begin{center}
					\begin{tabular}{ll}
					Eingereicht am:& 29.07.2019\\
					Betreuender Hochschuldozent:& Prof. Dr. Gefei Zhang\\
					Zweitgutachter: & Prof. Dr.-Ing. Kai Uwe Barthel\\
					\end{tabular}
				\end{center}
			\end{table} 
		\end{center}
		\vspace{0.6cm}
	%	{\bf Abstract}: Das Abstract
\end{titlepage}

\newpage
%\clearpairofpagestyles




%\clearpairofpagestyles

\pagenumbering{roman}
\setcounter{page}{10}

\section*{Abstract}\label{sec:abstract}
Im Zuge der Digitalisierung sind Bildungseinrichtungen mit  
neuartigen, nie dagewesen Problemen konfrontiert. 
Wie können sowohl bereits bewährte, als auch neue Unterrichtsmethoden sinnvoll durch digitale
Technik unterstützt werden? Wie kann der finanzielle Rahmen von zur Verfügung stehenden
Mittel optimal genutzt werden? Ist Datenschutz gewährleistet? 
\\ \\
Viele Anbieter setzen auf Cloud-Lösungen, welche eine Internetanbindung zur Nutzung obligatorisch macht. Schulen mit noch ausbaufähiger digitaler Infrastruktur 
benötigen jedoch skalierbaren Hard- und Softwarelösungen, die zur Not auch offline einsetzbar sind, um auch zeitnah kosteneffizient digitale Technik im Unterrichtsalltag zu integrieren.
\\ \\  
Im Rahmen dieser Arbeit wurde die gegenwärtige Situation in Sachen Digitalisierung hinsichtlich Soft- und Hardware sowie dem damit verbundenen Einsatz von Softwareanwendungen zu Unterstützung von interaktiven Unterrichtsmethoden analysiert. Anschließend darauf aufbauend ist eine Client-Server Web-Anwendung zur Lösung implementiert worden, welche in erster Linie offline funktioniert, wenig Anforderungen an die technische Infrastruktur stellt, aber sich auch in ausgebauter Umgebung sinnvoll nutzen lässt. 
\newpage

%\thispagestyle{empty}
\cfoot{\pagemark}
\tableofcontents 
\newpage

\pagenumbering{arabic}




%\pagestyle{scrheadings}
%\ihead{\leftmark}
%%\automark{section}
%\chead{\rightmark}
%%\automark{subsection}
%\ohead{\thepage}


\pagestyle{fancy}
\fancyhf{}
%\setlength{\headheight}{15pt}
\renewcommand{\headrulewidth}{0.4pt} %definiert die Dicke der Linie unter der Kopfzeile
%\renewcommand{\footrulewidth}{0pt} %definiert die Dicke der Linie über der Fußzeile
\renewcommand{\sectionmark}[1]{%
	\markboth{\thesection.\ #1}{}} %sorgt dafüpr, dass die Kapitel und Unterkapitel in der Kopfzeile stehen


%\addtolength{\headwidth}{\marginparwidth}
%\setlength{\fancyhead}{0.4\headrulewidth}
\fancyhead[L]{\textsl{\small 
%		\leftmark \hspace{0.8cm}
		\rightmark}}
%\fancyhead[C]{\parbox{0.5\textwidth}{\textsl{\small \rightmark}}}
\fancyhead[R]{\textsl{\small \thepage}}
%\fancyhead[L]{\parbox{0.3\textwidth}{\textsl{\small \leftmark}}}
%\fancyhead[C]{\parbox{0.6\textwidth}{\textsl{\small \rightmark}}}
%\fancyhead[R]{\parbox{0.05\textwidth}{\textsl{\small \thepage}}}
%\addtolength{\headwidth}{\marginparsep}
%\fancyfoot[C]{\thepage}


\thispagestyle{empty}
\section*{Selbstständigkeitserklärung}\label{sec:selbststandigkeitserklarung}
Ich erkläre hiermit, dass ich die vorliegende Bachelorarbeit selbstständig verfasst und dazu  keine  anderen  als  die  angeführten  Behelfe  verwendet,  die  Autorenschaft  eines Textes  nicht  angemaßt  und  wissenschaftliche  Texte  oder  Daten  nicht  unbefugt verwertet habe. Die elektronische Kopie ist mit den gedruckten Exemplaren identisch.
\vspace{5cm}
\\

Berlin, \today, 
\begin{flushleft}
	\line(1,0){350}\\
	(Ort, Datum, Unterschrift)
\end{flushleft}
\newpage

\input{sections/00Begriffe}
\newpage

\section{Einführung}\label{sec:einfuhrung}
\subsection{Motivation}\label{sec:motivation}

Am 04.04.2019 trat die Änderung des Art. 104c des Grundgesetz für die Bundesrepublik Deutschland in Kraft
und ebnete so den Weg für den von Bund und Ländern beschlossenen Digitalpakt Schule \cite{Art104cG55:online}. 
Dieser Beschluss macht deutlich, dass digitale Kompetenz im Bildungssektor von hoher Bedeutung ist, was auch von einer Förderungssumme von mindestens 5,5 Milliarden Euro unterstrichen wird. 
Legt man diese Summe auf die ca 40.000 Schulen um, erhält jede Schule einen Durchschnittsbeitrag von 137.000 Euro. Bei ca. 11 Millionen Schülerinnen und Schülern würde das eine Förderungssumme von ca. 500 Euro pro Schüler bedeuten. 
Einer der Hauptförderungspunkte des Digitalpakt Schule sieht den Ausbau der technischen Infrastruktur
an deutschen Schulen vor, z.B. Bereitstellung von drahtlosen Netzwerken, schnellen Internetzugangspunkten und digitale Unterrichtsmedien wie interaktive Whiteboards.
\\ \\
Das Bundesministerium für Bildung und Forschung (BMBF) argumentiert jedoch auch damit, dass kein digitales Medium alleine gute Bildung fördert, sondern immer dahinterstehende pädagogische Konzepte aus einer Vielfalt von Angeboten entscheidend sind \cite{dpakt2019:online} Dennis Horn (Experte für digitale Themen der ARD) kritisiert den zu starken Fokus auf Hardware und mahnt an, dass zu wenig darüber gesprochen wurde, wie diese denn auch sinnvoll genutzt werden kann. Man bereite keine Schülerinnen und Schüler auf eine digitale Welt vor, allein durch das Verteilen von Tablet-Computern\cite{Horn2018:online} \\ \\
Diese Problematik wurde auch auf der Podiumsdiskussion der re:publica 2018 - 'Was kommt in den digitalen Schulranzen?' angeschnitten. Tobias Hübner, Lehrer und Autor im Bereich Medienistik, zeigt dort ebenfalls auf, dass der Wille Geld auszugeben zu begrüßen sei, es aber an Konzepten und Materialien mangele. Als Lehrer würde er den Investitionsfokus auf Lehrerfortbildung setzen.
Es bestünde bereits eine Grundbereitschaft seitens vieler Lehrenden kleiner Ausgaben im Bereich Digitales selbst zu tätigen. Als ein Beispiel wäre hier der Einplatinencomputer Raspberry Pi zu nennen, welcher bereits für 33 Euro erwerblich ist (Stand April 2019) und genug Rechenkapazitäten bereitstelle um zahlreiche Projekte im Bildungsbereich durchzuführen. 
\\ \\
Im Vergleich kostet der populäre Tablet Computer 'iPad' der Firma Apple inc. in der günstigsten Variante bereits mindestens 449€ \cite{iPadmini65:online} (Stand April 2019), was schon knapp 90\% des Förderungsvolumens pro Schülerin und Schüler ausmachen würde.
\\ 

Seit dem Erfolgskurs des Web 2.0\footnote{Web 2.0 ist ein Schlagwort, das für eine Reihe interaktiver und kollaborativer Elemente des Internets, speziell des World Wide Webs, verwendet wird. Dabei konsumiert der Nutzer nicht nur den Inhalt, er stellt als Prosument selbst Inhalte zur Verfügung. - Wikipedia.org} in den frühen 2000er Jahren, zeichnet sich auch zunehmend der Trend des Software-as-a-Service Geschäftsmodells ab. Dies beschreibt die Bereitstellung von Software im Internet, ohne dass der Benutzende die Software selbst noch lokal installiert haben muss. Im Jahr 2015 setzten bereits über drei Viertel von 102 befragten Unternehmen Software dieser Form aktiv im Geschäft ein\cite{TecArt-GmbH2019:online} Dieses Phänomen, oft auch digitale Transformation betitelt, wurde nicht zuletzt durch die immer leistungsstärker werdenden Web-Technologien möglich. Viele Arten von Software können  mittlerweile in einer im Webbrowser lauffähigen Alternative substituiert werden. Ein populäres Beispiel ist die Office-Suite Google Docs der Firma Google inc. Hier lassen sich Textverarbeitung, Tabellenkalkulation und das erstellen von Präsentationen ohne Installation und direkt im Webbrowser des Benutzenden ausführen. Ein anderes Beispiel ist die Web-Software Photopea welche ebenfalls komplett im Web-Browser ausgeführt wird und dem nur lokal installiert ausführbaren quasi Industriestandard Bildbearbeitungsprogramm Photoshop der Firma Adobe inc. sehr nahe kommt. Im Vergleich zu lokal installierter Software ist die Bereitstellung von Web-Software einfacher, da solange ein moderner Webbrowser lauffähig ist, das Betriebssystem des Client-Computers zu vernachlässigen ist. Ebenso stellt potente Hardware keine zwingende Voraussetzungen, da etwaige rechenintensive Aufgaben auf der Serverseite getätigt werden können oder hier eine Balance zwischen Client und Server angestrebt werden kann. \\ 

Da ein o.g. Raspberry Pi Einplantinencomputer bereits genügend Leistung für Webtechnologien aufweist und einen sehr günstigen Anschaffungspreis aufweist, sowie bereits 67\% der 10-11 jährigen Jugendlichen ein Smartphone besitzen\cite{Statista2017:online} welche ebenfalls genug Leistung für Webanwendungen aufweisen, könnte der verstärkte Einsatz von Software, welche auf Webtechnologien basiert, die im Rahmen des Digitalpakt Schule fließenden Gelder optimierter ausschöpfen. 
%Hier jetzt Web Trend noch erwähnen!

\subsubsection{Besuch der Grundschule am Rüdesheimer Platz}
Im Rahmen der Vorrecherche zu dieser Arbeit wurde einem Unterrichtstag in 
einer Jahrgangsübergreifenden (JüL) Klasse 1 bis 3 an der Grundschule am Rüdesheimer Platz beigewohnt um ein differenzentierese 
Bild der gegenwärtigen Lern- und Digitalisierungssituation an einer Berliner Schule zu bekommen. An dieser Stelle eine große Dankaussagung an Frau Marie Wewer, Grundschullehrerin, welche diese Erfahrung möglich gemacht hat und in einem anschließenden Gespräch das Interesse an einer kostengünstigen und einfach nutzbaren Lösung zur Unterstützung von interaktiven Unterrichtsmethoden unterstrichen hat. Die Erprobung der im Rahmen dieser Arbeit implementierten Softwarelösung wurde ebenfalls an der Grundschule am Rüdesheimer Platz durchgeführt und wird im Kapitel \ref{sec:erprobung} erläutert.     

\newpage

\section{Grundlagen}\label{sec:grundlagen}
\subsection{Technik im Unterricht}\label{sec:technikunterricht}
\subsection{Exkurs (digital unterstützte) Interaktive Unterrichtsmethoden}\label{sec:interaktiveunterr}


% Vergleich zu anderen Modellen?

\subsection{Intranet und Internet}
% Unterschied und Gemeinsamkeit klar machen
% Hier kommunikation erklären! Protokolle und OSI schicht!
\subsubsection{Kommunikation}
\subsubsection{Client-Server Modell}\label{sec:clientservermodell}

\subsection{Überblick Webtechnologie}\label{sec:webbasedsoftware}
% Hier auf PDF Technische Anforderungen verweisen (Footnote?) da sehr ausführlich und gut
% Anfang Geschichtlich

% Begriff Internet
\subsubsection{World Wide Web}\label{sec:www}
%
\subsubsection{Webanwendungen}\label{sec:webanwendungen}
\
\subsubsection{Webservices}\label{sec:webservices}





\newpage

\section{Analyse}\label{sec:analyse}
In diesem Kapitel der Arbeit werden zunächst existierende Plattformen am Markt verglichen und darauf aufbauend Anforderungen an das Projekt formuliert.
\subsection{Vergleich mit existierenden Plattformen}\label{sec:vergleichplat}
Im folgenden Abschnitt werden ausgewählte existierende Plattformen, die im Bereich 
digital gestützte Unterrichtsmethoden angesiedelt sind, beleuchtet und anschließend 
gegenübergestellt. Eine klare Trennung zwischen kommerziell und nicht-kommerziell ist schwierig bis unmöglich, da viele Plattformen im Bereich des Freemium\footnote{Freemium ist ein Geschäftsmodell, bei dem das Basisprodukt gratis angeboten wird, während das Vollprodukt und Erweiterungen kostenpflichtig sind.} Geschäftsmodells vermarktet werden. Generell gibt es sehr viele Anbieter und Plattformen und somit ist eine Beschränkung der Auswahl unabdingbar. 
% es gibt sehr sehr viele tools, eine Beschränkung im Vergleich ist unabdingbar. 
% viele tools machen dinge verschieden. Beispiel single user prinzip (mind maps)

\paragraph{SMART Learning Suite Online}
Der Anbieter SMART (Smart Technologies Corporation) ist in Deutschland vor allem für sein Angebot von interaktiven Whiteboards, welche unter dem Namen SMART Board vermarktet werden, bekannt. Ergänzend bietet SMART auch die SMART Learning Suite an, welche sowohl online als auch als lokale Installation genutzt werden kann. Positiv hervorzuheben ist, dass bei der Cloud Variante der Nutzende vorab seine Server Region festlegt. Wird hier Europa gewählt, ist anzunehmen dass europäische Richtlinien im Bezug auf Datenschutz und Speicheranforderungen berücksichtigt werden. Gespeichert werden die Daten generell auf Amazon- oder Google-Servern, wobei Smart angibt, dass in der europäischen Service Region hierbei Amazon-oder Google-Server mit Standort Deutschland genutzt werden\cite{Technologies2019}. Die SMART Learning Suite kann sowohl online als auch offline installiert werden und kostenlos getestet werden. Getestet wurde jedoch nur die online Version, da nur diese im Webbrowser läuft, welches in Hinsicht auf dieses Projekt relevant ist. \\ Das Angebot umfasst viele Funktionalitäten und unterschiedlichste Implementierungen von  interaktiven Unterrichtsmethoden, wie Quiz/Befragungen, Brainstorming, Memory, Karteikarten u.v.m. Viele Anwendungen funktionieren im Einzelanwender-Betrieb, Lehrender, Schülerinnen und Schüler nutzen das gleiche Gerät. Andere Anwendungen erfordern zusätzliche Clients, sprich Geräte wie Smartphones oder Computer, laufen also im Mehrbenutzerbetrieb. Ebenso können Lehrende Prüfungsaufgaben erstellen und diese dann Abfragen und Auswerten. Eine strikte Unterscheidung zwischen Lehrer-, Klassen-, und Studierendenansicht findet nicht statt. Ein großer Anspruch der Software ist, dass ein Lehrender ein ganzes Set an Aktivitäten für den Unterricht erstellen kann und dieses schrittweise durchlaufen wird. Es kann bspw. mit einem Test begonnen werden, anschließend erfolgt ein Folie mit einem Begriff und darauffolgend wird ein interaktive Unterrichtsmethode ausgeführt usw. 
% Sehr cool und mächtig, doof beim brainstorming, begriff steht nicht in der mitte? Sehr klein alles! Nicht lokalisiert auf Deutsch

\paragraph{ClassFlow}
Ähnlich zu SMART Learning Suite Online ist ClassFlow eine Software, welche das 
Durchführen von interaktivem Unterricht ermöglicht. Anbieter ist  Der Lehrende erstellt hierzu Sitzungen, welche ähnlich einer Präsentation durchlaufen werden. An jeder Stelle kann der Lehrende den Bildschirminhalt an die Schülerinnen und Schüler Endgeräte schicken und interaktive Unterrichtsmethoden starten, welche z.B. Umfragen, Brainstormings, Kreuzworträtsel u.v.m. sein können. Die Software läuft in der Cloud, es existiert keine Offline Variante. 

% das ding mit den karten hochhalten seitens der Schüler
\paragraph{Pull Everywhere}
% Freemium modell auch, hier geben die Leute via app die antwort
% geteiltes Client prinzip (Presenter and teacher view)
% Gezielt für Umfragen, nicht quiz
\paragraph{Kahoot!}
% wohl kostenlos
\subsubsection{Gegenüberstellung}\label{sec:gegenstellung}
\subsection{Systembeschreibung}\label{sec:sysbeschreib}
Aus Gründen der Lesbarkeit wurde im folgenden Abschnitt die männliche Form gewählt, nichtsdestoweniger beziehen sich die Angaben auf Angehörige beider Geschlechter. 

Die Software soll als Web Server-Applikation implementiert sein. Über drei verschiedene Client Lösungen kann mit dem Server interagiert werden. \\ \\

1.) Über ein Backend Zugang können Administratoren und Dozierende den Server verwalten sowie erstmalig initialisieren. Neue Dozierende können einen Benutzeraccount anlegen, welcher von Administratoren freigeschaltet werden muss. Alternativ können Administratoren neue Benutzerkonten anlegen. Dozierende ist es möglich Lehreinheiten zu erstellen, diese zu starten sowie zu beenden. Während einer aktiven Lehreinheit, ist es Dozierenden möglich, diese zu leiten. (Fortschritt, Verbundene Schüler/Studenten verwalten, Speichern, je nach Typ der Lehreinheit.) Eine Lehreinheit ist eine interaktive Unterrichtsmethodik softwareseitig umgesetzt. Als erste Umsetzung erfolgt in diesem Projekt die Implementierung eines Brainstorming und Quiz. \\ \\

2.) Ein Präsenter Client zeigt die zur Laufzeit einer aktiven Lehreinheit relevanten Informationen an. Dieser ist zur Anzeige auf einem Großflächenanzeigegerät ausgelegt (Projektor, Fernseher, Smart Board). \\ \\
3.) Schüler/Studenten geben einen frei wählbaren Namen an. Ein Benutzerregistrierung ist nicht notwendig. Sie können anschließend aktiven Lehreinheiten beitreten und nach dem Start an diesen partizipieren. \\ \\

Eine detaillierte Aufführung der Anforderungen und Eigenschaften dieses Projekts erfolgt in den nachfolgenden Abschnitten. 

\subsection{Zielgruppe}\label{sec:zielgruppe}
Das Software-Projekt soll sich in erster Linie an Bildungseinrichtungen jeglicher Art und deren Dozierenden richten, welche eine lokal ausgeführte Softwarelösung für das Durchführen von interaktiven Unterrichtsmethoden bevorzugen. Darüber hinaus auch an jegliche, die digital gestützte interaktive Lern- und kompetitive Kleinstspiele nutzen möchten. Dabei ist eine flexible Skalierbarkeit des Servers gegeben (siehe Abschnitt \ref{sec:anforderung}). Desweiteren ist das Software-Projekt attraktiv für die Open-Source Community, welche das Projekt weiter ausbauen kann sowie neue Typen von Lehreinheiten (interaktive Unterrichtsmethode softwareseitig umgesetzt / 'Spiel'-Art) hinzufügen kann (weiterführend REF UNBEKANNT NOCH!.) 
\subsection{Abgrenzung}\label{sec:abgrenz}
Der Prototyp des Softwareprojekts (interne Bezeichnung Node ICT \footnote{Node ICT steht für Node.js interactive course teaching. Das Node.js Server Framework bildet den Grundstein des Softwareprojekts}) soll das Anlegen und Ausführen von Lehreinheiten ermöglichen. Der Prototyp wird auf dem lokalen Host getestet (Server und Client auf derselbe Maschine ausgeführt) sowie im Intranet (LAN, Server und Clients auf unterschiedlichen Maschinen ausgeführt). Eine verschlüsselte Kommunikation zwischen Server und Client ist erwünscht, wird jedoch nicht im Prototyp implementiert. Eine Nutzung über öffentlicher IP-Adresse oder Domain im Internet ist prinzipiell möglich, wird jedoch nicht getestet. Ebenso wird bei dem Prototyp vermindert Augenmerk auf das Design der Client Anwendungen gelegt, worunter die Usability jedoch nicht leiden soll. Das Design wird leicht anpassbar sein. Wie in Abschnitt \ref{sec:sysbeschreib} erwähnt, wird sich auf das Implementieren von zwei interaktiven Unterrichtsmethoden beschränkt, der Prototyp wird aber das Umsetzen und Hinzufügen weitere Unterrichtsmethoden ermöglichen, da prinzipielle nur die Logik der neuen Unterrichtsmethode geschrieben werden muss. Eine Wiederverwendbarkeit von grundlegender Funktionalität (z.B. Anbindung an Datenbank, Management verbundener Clients etc.) wird bereitgestellt.   
\subsection{Systemanforderungen}\label{sec:anforderung}
Aufbauend auf das Analysekapitel dieser Arbeit (Abschnitt \ref{sec:analyse} ff.) sowie dem Abschnitten \ref{sec:digianschulen} und \ref{sec:datenschutz} lassen sich funktionale und nicht-funktionale Anforderungen an das System ausformulieren, welche in den folgenden zwei Abschnitten gelistet werden.

\subsubsection{Funktional}\label{sec:anffunc}

\subsubsection{Nicht Funktional}\label{sec:nichtfunc}
\subsection{Technische Anforderungen}\label{sec:techanford}
In diesem Abschnitt werden die technischen Anforderungen an die Hardware erläutert, welche einen reibungslosen Betriebsablauf gewährleisten sollen. Es wird hierbei zwischen Server und Client Anforderungen unterschieden obgleich Server und Client auch auf der selben Maschine ausgeführt werden können. 
\subsubsection{Server}\label{sec:servertech}
Die Server Software soll so umgesetzt werden, dass sie auch auf leistungsschwächerer Hardware problemlos mehrere Benutzer gleichzeitig ohne signifikante Performanceeinbuße bedienen kann. Die Hardwarespezifikationen eines Raspberry Pi (Version 3B) Einplantinencomputer (Querverweis zu Abschnitt \ref{sec:einfuhrung}) soll hierbei als Mindestanforderung definiert sein. Durch die Nutzung des Serverwebframeworks 'Node.js' als Grundgerüst der Software, ist ein plattformunabhängiger Betrieb gewährleistet. Der Server soll rein im Intranet lauffähig sein und keine online Abhängigkeiten besitzen. Daher soll technisch keine (Breitband)Internetanbindung für den Betrieb notwendig sein. Es ist ebenso soll es möglich sein den Server 'headless', d.h. ohne angeschlossene Peripheriegeräte wie Tastatur, Maus und Bildschirm zu betreiben. Die Initialisierung der Software kann hierbei durch ein Startskript oder beispielsweise über einen SSH\footnote{Mittels SSH (Secure Shell) kann eine verschlüsselte Verbindung zur Kommandozeile (Shell) auf einem Server hergestellt werden} Zugang erfolgen.
\subsubsection{Client}\label{sec:clienttech}
Der im Rahmen dieses Projekts zu entwickelnde Client Software kann wie in Abschnitt \ref{sec:sysbeschreib} erläutert, in drei Parts eingeteilt werden. Die Verwaltung im Backendbereichs der Server Software soll eine besonders niedrige Hardwareanforderung aufweisen, da hier gänzlich auf den Einsatz von JavaScript verzichtet werden wird. Dies soll eine Server-Verwaltung auch bei deaktiviertem JavaScript gewährleisten. Die restlichen Software Module sollen in jedem modernen Webbrowser auf jedem Endgerät lauffähig sein. Der Einsatz von JavaScript ist hier unverzichtbar. Eine Kompatibilitätsabdeckung von 95\% zur ECMAScript 6 (ECMAScript 2015) Spezifikation sollte vom verwendeten Webbrowser gegeben sein. Diese Anforderung erfüllen jedoch alle modernen Webbrowser (Stand 2019)\cite{ECMAScri7:online}. 

\newpage

\section{Konzept}\label{sec:konzept}
Das folgende Kapitel soll Gedanken zur Realisierung des Projekts widerspiegeln. 
Dies umfasst den allgemeinen Systemaufbau, sowie Entwürfe und Architekturvorstellungen hinsichtlich der Server- und Clientseite und die damit verbunden eingesetzten Softwaremodule.
\subsection{Systemaufbau}\label{sec:sysaufbau}
Das System soll in eine Server- und Clientseite aufgeteilt sein. 
Da es sich um eine Webapplikation handeln soll, wird die Interaktion mit dem Server über einen Webbrowser stattfinden und dieser soll auch gleichzeitig der Client sein. Für eventuelle Wartungsaufgaben und Aufgaben wie das Starten des Servers, soll die Steuerung über eine Kommandozeile möglich sein. Alle anderen Interaktionen werden über den Client ausgeführt.
\subsection{Netzwerkaufbau}\label{sec:netzwerkaufbau}
Als Kommunikationsprotokoll soll das aus dem Webbereich bekannte HTTP resp. HTTPS Protokoll zum Einsatz kommen. Der Server wird als Webserver fungieren, Anfragen müssen vom Client aus initiiert werden (vgl. Abschnitt \ref{sec:www}). Bei Parts, welche Echtzeitinteraktion benötigen, soll das Websocket (ws) Protokoll zum Einsatz kommen, welches eine bidirektionale Kommunikation zwischen Server und Client ermöglicht. Der Server soll sowohl in einem Intranet wie auch im Internet lauffähig sein. Dabei kann er, je nach infrastruktureller Realisierung über ein IPv4 oder IPv6 Adresse erreicht werden, bei Nutzung eines DNS-Servers auch über eine Domain.\\ \\
Schlussfolgernd aus Systemaufbau und Netzwerkaufbau lässt sich folgende Abbildung skizzieren:

\begin{figure}[H]
	\centering
	\includegraphics[width=0.8\linewidth]{bilder/server_diagram}
	\caption[Kommunikationsaufbau des zu entwickelnden Systems]{Kommunikationsaufbau des zu entwickelnden Systems}
	\label{fig:server_diagram}
\end{figure}


\subsection{Entwurf des Servers}\label{sec:serverkonzept}
Aufbauend auf das Grundlagen und Analysekapitel sollen in diesem Abschnitt die Entwurfsgedanken hinsichtlich der Serverkomponente des Projekts widergespiegelt werden. 
\subsubsection{Laufzeitumgebung: Node.js}\label{sec:nodejs}
Als Laufzeitumgebung und Grundbaustein des Servers soll die JavaScript-Laufzeitumgebung Node.js genutzt werden, da dies zwei wesentliche Vorteile mit sich bringt:
\begin{enumerate}
	\item Node.js nutzt als Paketmanager und Projektverwaltungstool \textbf{NPM} (Node Paket Manager). Mit dieser Software ist der Zugang zu über 350.000 Paketmodulen (Stand 13. Januar 2017) gegeben und diese können das Projekt Modular erweitern. Ebenso können mit NPM grundlegende Start- und Installationsskripte leicht ausgeführt werden. 
	\item Da Node.js eine JavaScript-Laufzeitumgebung ist, wird zur Programmierung die Skriptsprache JavaScript genutzt, welche auch auf der Clientseite im Webbrowser zum Einsatz kommt. Dies erleichtert den Implementierungsprozess, da einheitlich in einer Sprache geschrieben wird.
\end{enumerate}
Darüber hinaus können NPM-Pakete auch auf der Clientseite genutzt werden (siehe dazu auch Abschnitt \ref{sec:browserify}). Eine gute Skalierbarkeit ist ebenfalls gegeben. Dies wird in folgenden Abschnitten genauer erläutert. Ist ein besonders hohen Ressourcenbedarf von Nöten (z.B. eine Bildungseinrichtung möchte einen zentralen Server installieren, welche viele Klassen/Kurse bedienen soll) können mehrere Serverinstanzen auf einer Maschine parallel laufen und vorab mit einem Lastenverteiler (Load Balancer) Server, wie z.B. NGINX verwaltet werden. Durch die genannte Argumentation soll das Projekt mit NodeJS realisiert werden und nicht mit einem php-Framework. \\ Da Node.js grundlegend sehr offen ist was seinen Einsatzzweck betrifft, soll als Webserver Modul das Node-Paket \textbf{Express.js} genutzt werden, welches im nächsten Abschnitt genauer erläutert wird.
\subsubsection{Webserver: Express}\label{sec:expressjs}
Um mit Node.js komfortabel eine Webapplikation zu implementieren, soll das bekannte Web-Framework Express eingesetzt werden, welches viele HTTP-Dienstprogrammmethoden und den Einsatz von Middlerwarefunktionen gestattet. Hierbei wird jeder eingehende HTTP Request von Funktion zu Funktion weitergeleitet (Aufruf der Methode \texttt{next()}) oder explizit beantwortet (Die Funktion besitzt ein Rückgabewert). Ebenso ist mit Express das Abbilden von Routen möglich. Express soll für den gesamten administrative Teil des Lehrer-Login zum Einsatz kommen. Ebenso soll durch Express das Anlegen und Editieren von Lehreinheiten möglich sein (vgl. \ref{sec:sysbeschreib}). 
Da für den gesamten Lehrer-Backendbereich Express zum Einsatz kommen soll und hier mit einfachen HTTP-Requests gearbeitet wird, kann auf der Einsatz von JavaScript auf der Client-Seite auf ein Minimum reduziert werden, was den Einsatz auf Servereinheiten mit nicht modernem Webbrowser entgegenkommt (Sollte Client und Server auf der gleichen Maschine ausgeführt werden).\\ \\ 
Für den interaktiven Part des Projekts sollen zur Kommunikation WebSockets genutzt werden, welche mithilfe der JavaScript-Bibliothek \textbf{Socket.IO} realisiert werden sollen. Dies wird in der nächsten Sektion beschrieben.   

\subsubsection{SocketIO}\label{sec:socketio}
Die JavaScript-Bibliothek Socket.IO ermöglicht bidirektionale Echtzeit-Kommunikation zwischen Webclient und Server, wobei dabei Bibliothek sowohl auf Server- wie auch Clientseite zum Einsatz kommt. Ein großer Vorteil ist, dass beide Komponenten eine nahezu identische API aufweisen. Daten können sehr einfach von Client ereignisgetrieben (event-driven) zwischen Server und Client sowie vice versa ausgetauscht werden. Client und Server lauschen dabei gegenseitig auf Ereignisse, wie das Verbinden eines neuen Clients oder auch selbst implementierte Ereignisse. Dabei können jegliche JavaScript Daten hin-und hergeschickt werden. Eine händische Konvertierung in das JSON-Format ist nicht notwendig. Für das Anmelden von Schülerinnen und Schülern, das Durchführen von interaktiven Unterrichtsmethoden soll SocketIO zum Einsatz kommen. Hierzu soll Express die entsprechenden Client Daten auf einer festgelegten Route senden und anschließend die Kommunikation von SocketIO kontrolliert werden. Da die Nutzung der Software rein im Intranet nutzbar ist und Nutzende über ein WLAN Zugriff erhalten können, ist der im Abschnitt \ref{sec:websockets} genannte Nachteil von erhöhtem Datenverkehr zu vernachlässigen.      
\subsubsection{Sonstige Module}
Neben Express ist der Einsatz von weiteren Modulen (Node Packages) vorgesehen, welche unterschiedliche Funktionalitäten realisieren sollen. Diese sind:
\begin{itemize}
	\item \textbf{Body-Parser}: Diese Modul ermöglicht das einfach Auslesen von HTTP-Requests möglich. Schickt ein Client bspw. Formulardaten können diese einfach gelesen und ausgewertet werden. Dies soll im Backendbereich der Lehrenden oftmals die Praxis sein.
	\item \textbf{express-session}: Da Lehrende und Administratoren zur Nutzung der Software einen gültigen Zugang besitzen müssen, ist zur Authentifizierung des Nutzenden der Einsatz von Sessions vorgesehen (Querverweis Abschnitt\ref{sec:www}). Das Modul Express-Session macht das Arbeiten mit diesen sehr komfortabel. Über das Zusatzmodul \texttt{connect-session-sequelize} ist die Zusammenarbeit mit der gewählten Datenbank einfach. (Weiterführende Informationen diesbezüglich im Abschnitt  \hyperref[sec:datenbank]{Wahl der Datenbank}). 
	\item \textbf{Pug}: Die Template Engine Pug besitzt seine eigene Syntax und macht das Entwerfen und Schreiben von HTML Templates, welche serverseitig übermittelt werden, sehr komfortabel. Zusätzlich werden Funktionalitäten wie Vererbung und Mixins unterstützt. Ein Einsatzbeschreibung erfolgt in Kapitel \hyperref[sec:implementierung]{Implementierung}. Pug soll für sämtliche zu übertragende HTML Dokumente zum Einsatz kommen.   
\end{itemize}
\subsubsection{Wahl der Datenbank}\label{sec:datenbank}
Da bei dem zu entwickelnden System vielerlei Daten anfallen, wie registrierte Nutzer, angelegte Kurse, interaktive Unterrichtseinheiten und mehr, ist der Einsatz eines Datenbanksystems unerlässlich. Grundlegen können Datenbanksysteme in zwei Kategorien unterteilt werden: \\
\textbf{SQL} und \textbf{noSQL} Systeme. \\ SQL Systeme speichern ihren Daten in sogenannten Relationsmodellen, welche als Tabelle visualisiert werden können. Hierbei beschreibt der Tabellenkopf den Datensatz und seine Datentypus (jede Spalte für sich) während Zeilen eine Entität (Eintrag) in der Datenbank beschreiben. Vorteil hierbei ist, dass die Daten konform sind, d.h. bei Zugriff liefert immer einen Rückgabewert \cite{neumann2015entwicklung}. Nachteil ist der erhöhte Aufwand, sollte die Definition des Relationsmodells im Nachhinein geändert werden, was das Aktualisieren sämtlicher Daten erfordern würde. Desweiteren sind SQL System schwer skalierbar, da für größere Datenbanksysteme potentere Server gekauft werden müssen. Mehrere Relationsmodelle können über Fremd-Schlüsse (Querverweise) miteinander verbunden werden, um auch komplexere Sachverhalte abbilden zu können. \\ \\ NoSQL lassen sich in verschiedene Subkategorien je nach Arbeitsweise beim Speichern der Datensätze einteilen\cite{neumann2015entwicklung}. Am populärsten sind Dokumentenorientiert, Key-Value Pairs (Schlüssel-Wert Paare) und Graphen-basierte Systeme. Dokumentenorientierte NoSQL Datenbanken legen pro Entität ein Dokument an, in welchem die Informationen meist im JSON Format abgespeichert werden. Key-Value Systeme verfolgen ein einfaches Zuordnungsprinzip und bilden Schlüssel-Wert Paare, ähnlich einer Dictionary Datenstruktur. Bei Graphen-basierten Systemen werden Entitäten und ihre Beziehungen untereinander an sich gespeichert. Generell sind NoSQL Systeme weniger statisch definiert im Vergleich zu SQL Systemen. Dies räumt eine große Flexibilität beim Speichern von Daten ein, da Datensätze auch unvollständig gespeichert werden müssen. Dies kann auch als Nachteil interpretiert werden, ist aber generell immer vom Kontext des Projekts abhängig. \\ \\    

Für das zu entwickelnde System soll ein möglichst flexibler Weg gewählt werden was die Wahl der Datenbank betrifft. Da das MVC-Prinzip zum Einsatz kommen soll, beschreibt der Modell Teil von zu bereitstellenden Daten auch wie diese über welche Funktionalität aus der Datenbank geladen werden sollen. Den Controller soll nur die vom Modell bereitgestellten Funktionalitäten nutzen und keine direkten Datenbankzugriffe selbst ausführen. Damit die Software im hohem Maße skalierbar bleibt, ist der Einsatz eines sogenannter Object-Relationship-Mapper, kurz ORM, (Objektrelationale Abbildung) vorgesehen, der an verschiedenste Datenbanksysteme angebunden werden kann. Da das Projekt in seiner kleinsten Skalierung lokal auf einem Einplantinencomputer wie dem Raspberry Pi 3 und im lokal im Intranet laufen können soll, ist für den Anfang die Verwendung eines Datenbanksystems, welches vollständig durch eine Programmbibliothek lauffähig ist, vorgesehen. Dies hat den Vorteil, dass kein extern laufendes Datenbanksystem installiert, gewartet und gestartet werden muss, da die komplette Datenbank in einer einzigen Datei auf dem Server gespeichert wird. Diese Anforderungen erfüllt die gemeinfreie Programmbibliothek \textbf{SQLite}. Die gesamte Datenbank kann hier sogar rein im Arbeitsspeicher gehalten werden, was jedoch den Nachteil mit sich bringt, dass bei einem Ausfall oder Abschalten des Server der kompletten Verlust sämtlicher Daten mit einhergeht. 
\\ Als ORM fiel die Wahl auf das Node.js Modul Sequelize, welches neben SQLite mit viele andere bekannte SQL Datenbanksystemen zusammenarbeiten kann, u.A. Postgres, MariaDB und Microsoft SQL Server. Der Wechsel auf ein anderes SQL Datenbanksystem ist somit jederzeit problemlos möglich, falls gewünscht. \\ \\  

Das Zusatzmodule \texttt{connect-session-sequelize} ermöglicht eine einfache Handhabung der Session-Verwaltung von eingeloggten Lehrenden in das System. Dazu werden entsprechende Tabellen zur Verwaltung der Sessions und deren Lebenszeit automatisch in der Datenbank  via Sequelize angelegt. Zuvor sollen die Passwörter sicher gespeichert werden, d.h. nicht im Klar-Text, nur als Hashwerte welche zusätzlich mit einem Salt verstärkt werden\footnote{Weiterführende Informationen unter:  \url{https://de.wikipedia.org/wiki/Salt_(Kryptologie)}}.

Da zum Zeitpunkt der Recherche kein zu SQLite ähnliches und für den produktiven Einsatz bereites NoSQL Äquivalent gefunden werden konnte, fiel die Wahl auf SQLite. Die genannten Vorteile eines NoSQL Systems scheinen für die Anforderungen des zu entwickelenden Systems ohnehin nicht relevant, obgleich sogar ein Umstieg auf NoSQL Datenbkansystem möglich wäre, auch wenn dies mit einem etwas erhöhten Aufwand einhergeht, da dann auch der ORM gewechselt und die Modelle entsprechend angepasst werden müssten.  

\subsubsection{Serverarchitekturdiagramm}\label{sec:serverarchitekt}
Schlussfolgernd lässt sich der finale Entwurf des Servers folgend visualisieren und wird in Kapitel \textbf{\ref{sec:implementierung} Implementierung} umgesetzt.

\begin{figure}[H]
	\centering
	\includegraphics[width=0.8\linewidth]{bilder/server_architektur}
	\caption[Aufbau der geplanten Serverarchitektur]{Aufbau der geplanten Serverarchitektur}
	\label{fig:server_diagram}
\end{figure}
\footnotesize{Hinweis: NodeJS Module aus der Sektion \textbf{Sonstige Module} wurden aus Gründen der Visualisierung in dieser Grafik nicht näher betrachtet. }

\subsection{Entwurf des Clients}\label{sec:clientkonzept}
Wie zuvor erörtert, ist es vorgesehen drei verschiedene Clients zu implementieren, welche alle auf dem gleichen Prototyp basieren sollen, allerdings verschiedene Zwecke verfolgen. Dies sieht ein Webclient jeweils für \textbf{Lehrende/Dozierende}, \textbf{Schülerinnen und Schüler} und einen für \textbf{Großbildanzeigen} optimierten wie Projektoren o.Ä. vor. Zur Vereinfachung werden diese gemäß der vorherigen Reihenfolge \textbf{Teacher Client}, \textbf{Student Client} und \textbf{Presenter Client} in diesem und darauffolgendem Kapitel genannt.\\ \\ 

\subsubsection{Gedanken zu UI}\label{sec:uientwurf}
// TODO am CSS Tag schreiben!
\subsubsection{Browserify}\label{sec:browserify}
Um das Nutzen von Node.js Packages sowie damit verbundene \texttt{require()} Funktionalität auch auf der Clientseite zu ermöglichen, soll die JavaScript Bibliothek \textbf{Browserify} zum Einsatz kommen. Mit ihr können alle Node.js Packages, welche über den Node Package Manager in das Projekt hinzugefügt wurde im Webbrowser des Clients genutzt werden. Dazu bündelt die Bibliothek alle Module und stellt anschließend eine einzige JavaScript Datei zur Verfügung, die nur noch im HTML-Dokument eingebunden werden muss. Mithilfe der \texttt{require()} Funktionalität, welche von Node.js gestellt wird, kann der Code auch übersichtlicher in mehrere Dateien/Module aufgeteilt werden. Dies war ohne Aufwand auf der Browserseite nicht ohne weiteres möglich, ist aber durch die Einführung von Modulen seit ECMAScript\footnote{Der als ECMAScript (ECMA 262) standardisierte Sprachkern von JavaScript beschreibt eine dynamisch typisierte, objektorientierte, aber klassenlose Skriptsprache. (Wikipedia.org)} in Version 6  nun möglich. Oftmals wird aber aus Kompatibliltätsgründen  zu älteren Webbrowsern auf Lösungen wie Browserify oder auch Babel gesetzt. Letztere übersetzt den geschriebenen JavaScript Code so, dass er auch von älteren Webbrowsern interpretiert wird (auch Transpiler genannt). Sämtliche folglich genannten JavaScript Module resp. Bibliotheken sollen via Browserify in eine JavaScript Datei zusammengefasst werden und anschließend pro Client eingebunden werden. Das bringt den zusätzlichen Vorteil dass für jegliches, auf Browserseite eingesetztes JavaScript nur einen einzigen HTTP-Get Request benötigt, da wie erwähnt nur eine JavaScript Datei pro Client angefordert werden muss. 
\subsubsection{JavaScript Lösungen}\label{sec:clientjs}
Von folgenden JavaScript Lösungen soll auf der Clientseite Gebrauch gemacht werden, um den Implementierungsprozess zu optimieren:
\begin{itemize}
	\item \textbf{VueJS}: Um das Anzeige, Editieren und Anpassen dynamischer Inhalte zu erleichtern, soll das JavaScript-Webframework VueJS zum Einsatz kommen, da dieses gut skalierbar ist und alle benötigten Funktionalitäten mit sich bringt. Im Vergleich zu AngularJS und React (siehe auch Abschnitt \ref{sec:clientseitigeransatz}), biete VueJS eine flachere Lernkurve und kann als ein guter Kompromiss aus seinen zwei Konkurrenten betrachtet werden. Mittlerweile hat VueJS seinen Konkurrent React in Sachen Popularität auf GitHub überholt\cite{Daityari2019}. VueJS ist auch gemessen an der Dateigröße von nur 80 KB deutlich kleiner im Vergleich zu Angular mit 500 KB.
	\item \textbf{SocketIO}: Das bereits in Abschnitt \ref{sec:socketio} erwähnte SocketIO besitzt ein Gegenstück, welches auf der Clientseite im Webbrowser zum Einsatz kommt. Dadurch wird die bidirektionale Kommunikation mit der Server ermöglicht und es soll in beide Richtungen Daten in Echtzeit ausgetauscht werden. 
	\item \textbf{Zingchart}: Zur Visualisierung der Wörterwolke (WordCloud), welche bei der interaktiven Unterrichtsmethode Brainstorming zum Einsatz kommt, und sonstigen Diagrammen soll die auf diese Szenarien spezialisierte JavaScript Bibliothek ZingChart genutzt werden. Die Software ist in einer freien Version erhältlich, wobei lediglich ein kleines ZingChart Logo stets angezeigt werden muss\cite{zingchartpricing}.
\end{itemize}

\subsubsection{Lehrer - Bereich}\label{sec:lehrerbereich}
\subsubsection{Schüler - Bereich}\label{sec:schuelerbereich}
\newpage

\section{Implementierung}\label{sec:implementierung}
%TOTO LEFTOVER: 
%QUIZ
%HTTPS
%WLAN
Das folgende Kapitel wird Einblicke in die Entwicklung der Software im Rahmen des Projekts geben. 
Dabei wird iterativ chronologisch die Vorgehensweise schriftlich reflektiert und an mehreren Stellen 
zum besseren Verständnis auch ein Einblicke in den Sourcecode gegeben. Anknüpfend werden etwaige Probleme bei der Implementierung aufgezeigt sowie mögliche Lösungen diskutiert. 

\subsubsection{Implementierung der Server-Software}\label{sec:implementserver}
Aufbauend auf das Entwurf Kapitel soll die Server-Software mit NodeJS und ExpressJS als Hauptkomponente entwickelt werden. Dazu wird zunächst im folgenden Abschnitt \ref{sec:implementexpress} der HTTP Server grundlegend konfiguriert und anschließend dessen Routen im darauffolgenden Kapitel \ref{sec:anlegrouten} angelegt. 

\subsubsection{ExpressJS Setup}\label{sec:implementexpress}
Nachdem das Projekt grundlegend mit dem Befehl \texttt{npm init} initialisiert wurde,
kann ExpressJS einfach über den Node Packet Manager (nachfolgend NPM genannt) hinzugefügt werden. 
Zusätzlich wird das NPM Paket IP genutzt um die aktuelle IP-Adresse der Maschine zu automatisch ermitteln und
den ExpressJS Server auf dieser lauschen zu lassen. Dies ist mit wenigen Zeilen Code erledigt:
\begin{lstlisting}[caption=Errichtung des Webservers]
const app = express();
const server = app.listen(3000, server_ip, function () {
logger.log({ level: 'info', message: `Hello! The Server is running on ${server_ip}!`});
});
\end{lstlisting}
Der Server lauscht auf der IP Adresse des Adapters der Maschine auf dem er ausgeführt wird, zusätzlich auf Port 3000, dies sollte je nach Konfiguration auf den Standard HTTP Port 80 resp. 443 geändert werden, sollte Verschlüsselung eingerichtet sein (HTTPS).
\\
Anschließend können nun die Routen eingerichtet werden.

\subsubsection{Autarkes WLAN}\label{sec:eigeneswlan}
Um einen Stand-Alone Betrieb\footnote{Stand-Alone meint einen Betrieb unabhänging von ggf. vorhandenen Netzwerkinfrastruktur am Einsatzort} der Software zu ermöglichen, wird als Lösung der Betrieb eines unabhängigen kabellosen Netzwerkes (WLAN/Wifi) angestrebt. Folgende mögliche Lösungsszenarien wurden ausgearbeitet.
\begin{enumerate}
	\item \textbf{Einfach}: Als einfachste Lösung hat sich der Betrieb eines lokalen WLANs über ein Smartphone oder Laptop herausgestellt. Nahezu jedes Smartphone bzw. jeder Laptop lässt das generieren eines WLAN Zugangspunkt für andere Geräte zu. Der Server wird mit diesem Netzwerk verbunden, sowie alle Clients. Dies wurde getestet und ein Betrieb war möglich. Da hierbei aber auch die Internetverbindung des Zugangspunktes freigegeben wird, was ggf. unerwünscht sein kann, würde es sich anbieten eine spezielle 'Companion App' zu entwickeln und für Android / iOS basierte Smartphones zu entwickeln. Diese könnte automatisch einen WLAN-Hotspot erstellen und den Netzwerkverkehr eventuell limitieren. Gleiches gilt analog für windows- oder unixbasierte Computer, ist aber problematischer, siehe dazu nächsten Listenpunkt.
	\item \textbf{Speziell}: Um auf einem Computer vollautomatisch einen WLAN Hotspot zu genieren, bedarf es Administrator Privilegien sowie genauere Kenntnisse über den verwendeten Netzwerkadapter. NodeJS selbst bietet nur eingeschränkt Möglichkeiten an, diese Aufgabe autark zu übernehmen, könnte aber ggf. eventuelle Shell-Skripte triggern. Unter Microsoft Windows 7+ gibt es mit den NPM Paket \texttt{node-hotspot} auch die Möglichkeit, dies direkt mit NodeJS zu erledigen. Diese Paket wird in die zu entwickelnde Software integriert und soll anschließend unter einer Microsoft Windows Umgebung das generieren eines WLAN Hotspots / Zugangspunktes ermöglichen. Zur Verwendung werden entsprechende Steuerungsoptionen in Einstellungsbereich im Lehrkraft Backend integriert. Die interne Steuerung erfolgt über Routen (siehe auch Abschnitt \ref{sec:anlegrouten}).
	
Wird die Applikation in einer vorhandenen Netzwerkinfrastruktur betrieben, ist ein Betrieb in jedem Falle gewährleistet. Ein mobiler WLAN-Router könnte ebenfalls genutzt werden, sollte keine ausreichende Infrastruktur vorhanden sein. Dieser müsste einmalig konfiguriert werden und ist anschließend in der Lage die Serverapplikation für Clients ansprechbar zu machen. Ein solches Gerät gibt es bereits ab ca. 10 Euro zu erwerben.
\end{enumerate} 

\subsubsection{Verschlüsselung}\label{sec:encrypted}
Um eine verschlüsselte Kommunikation zwischen Server und Clients zu gewährleisten, ist der Datenaustausch über das HTTPS Protokoll vorzuziehen. Um HTTPS allerdings sinnvoll nutzen zu können, ist ein Zertifikat von einer Zertifizierungsstelle (CA) notwendig, was meist mit Kosten verbunden ist. Allerdings bietet der Anbieter \textbf{Let's Encrypt}\cite{LetsEncrypt.org} kostenlose Zertifikate an, welche sich problemlos bei vorhandenem SSH-Zugang installieren lassen. Für den Intranet Betrieb können relativ einfach eigens ausgestellte Zertifikate genutzt werden, welche mit Shell-Programmen wie \textbf{openssl} generierbar sind\cite{Copes2018}. Allerdings warnen moderne Webbrowser den Nutzenden recht auffällig, dass die genutzte Verbindung dennoch nicht sicher ist, da dem Zertifikat nicht vertraut werden kann. Der HTTPS Betrieb wurde erfolgreich getestet. Die Implementierung ist nicht aufwendig allerdings, birgt aber den o.g. Nachteil. Da das Intranet ein an sich abgeschlossenes Netzwerk ist, scheint der verschlüsselte Betrieb in diesem zunächst nicht wichtig, kann aber jederzeit realisiert werden und ist bei Betrieb im Internet als obligatorisch zu betrachten. 

\subsubsection{Anlegen der Routen}\label{sec:anlegrouten}

Grundlegend soll es folgende Routen auf dem Server geben:
\begin{itemize}
	\item \textbf{'/'}: Die Haupt Route, sie wird angesteuert, wenn der Server einfach unter seiner IP (oder eingerichteter Domain) kontaktiert wird. Hier wird anschließend die Rolle des Nutzers (Lehrender oder Schülerin/Schüler) abgefragt  und dementsprechend weitergeleitet.
	\item \textbf{'/teacher'}: Diese Route verweist auf den Lehrerbereich der Anwendung, man kann sie auch als Backendbereich bezeichnen. Alle hinter dieser Route liegende Routen bedürfen einer Authentifizierung seitens des Nutzenden.
	\item \textbf{'/client'}: Diese Route verweist grundlegend auf den Student Client. Aber auch der Presenter Client wird über diese Weiche aufgerufen. 
\end{itemize}
Neben diesen drei Hauptrouten existieren noch weitere spezial Routen für das Error-Handling (z.B. 404 - Seite nicht gefunden) sowie besondere für die WebSocket Kommunikation, welche aber im Hintergrund genutzt werden und im Abschnitt \ref{REF UNKNOWN YET!} beleuchtet werden.\\ 
Das Anlege der Routen ist mit folgendem Codeausschnitt durchgeführt:
\begin{lstlisting}[caption=Anlegen der Routen]
app.use('/teacher', teacherRoutes);
app.use('/client', clientRoutes);
app.use('/', mainRoutes);
\end{lstlisting}
\footnotesize{Die Route Module werden im Hauptmodul (app.js) referenziert und deren Zuständigkeit festgelegt.}
\normalsize
Zu beachten gilt: Die weiterführend Routen werden in Dateien ausgelagert, um die Übersicht des Quelltextes zu wahren. Ebenso sind auch diese Routen sog. Middleware-Funktionen. Dies wird im nächsten Abschnitt genauer beleuchtet. Daher ist auch die Reihenfolge wichtig, würde die \texttt{'/'} Route als erstes angelegt werden, so würde diese alle folgenden, spezifischeren 'abfangen'.  


\subsubsection{Reflexion des MVC Schemas}\label{sec:mvc}
Da der Server nach dem Model-View-Controller Muster grundlegend arbeiten soll, gilt es dieses zu implementieren. Die folgende Tabelle soll ein Überblick über die anfallende Struktur geben:

\begin{table}[h!]
	\caption{MVC Struktur der Implementierung}
	\label{tab:mvcschema}
	\begin{tabular}{|l|l|l|l|}
		\hline
		\multicolumn{1}{|c|}{\textbf{Betrifft}}                         & \multicolumn{1}{c|}{\textbf{Model}} & \multicolumn{1}{c|}{\textbf{View(s)}}                                                                         & \multicolumn{1}{c|}{\textbf{Controller}}                                            \\ \hline
		Lehrende                                                        & user.js                             & \begin{tabular}[c]{@{}l@{}}teacher/new.pug\\ teacher/signup.pug\\ teacher/user-edit.pug\end{tabular}          & teacher.js                                                                          \\ \hline
		\begin{tabular}[c]{@{}l@{}}Schülerinnen/\\ Schüler\end{tabular} & student.js                          & student.pug                                                                                                   & client.js                                                                           \\ \hline
		Lehreinheiten                                                   & eduSession.js                       & \begin{tabular}[c]{@{}l@{}}edusessions/*/*.pug\\ edusessions/index.pug\\ edusessions/running.pug\end{tabular} & \begin{tabular}[c]{@{}l@{}}session.js\\ quizzing.js\\ brainstorming.js\end{tabular} \\ \hline
	\end{tabular}
\footnotesize {
	Zum Zwecke der Übersicht wurden einige interne Controller-Dateien nicht gelistet, wie z.B. der Error-Controller, welcher zwar eine View besitzt, jedoch kein Model. 
}
\end{table}
\newpage
Folglich müssen die entsprechenden Controller-Funktionen an Routen gebunden werden. Dabei wurde sich teils am RESTful Design orientiert, die Umsetzung erhebt jedoch kein Anspruch vollkommen 'RESTful' zu sein. Zunächst müssen entsprechende Unterrouten zu den aus Abschnitt \ref{sec:anlegrouten} bereits angelegten hinzugefügt werden. Zwecks der Übersicht wird hierzu ein Routes Ordner angelegt, welche die entsprechenden Router enthalten soll. 
Folgende Router werden angelegt: \\ \\
\textbf{routes/client.js}: Legt alle Student Client und Presenter Client relevanten Routen fest.\\
\textbf{routes/teacher.js}: Alle für das Backend resp. Lehrerbereich relevanten Routes werden hier angelegt.\\
Es folgt ein Codebeispiel aus der Datei routes/client.js. 
\begin{lstlisting}[caption=Unterrouten und Controlleranbindung]
// 3rd Party Imports
const express = require('express');
const router = express.Router();
// App Imports
const clientController = require('../controllers/client');
const isAuth = require('../middleware/is-auth');
// Presenter & Student Clients
router.get('/presenter/:sessionId', isAuth, clientController.getPresenter);
router.get('/student', clientController.getStudent);
router.get('/', clientController.getStudent);
module.exports = router;
\end{lstlisting}
Zur Verdeutlichung wird anknüpfend die in Zeile 10 des vorangegangen Listings Funktion \texttt{getStudent} des Controllers gezeigt:
\begin{lstlisting}[caption=GET Funktion des Student Controllers]
// GET => /client/student
exports.getStudent = (req, res, next) => {

return res.render('client/student',
	{
		docTitle: 'Student | Node ICT',
		ipAdd: ip.address(),
	});
};
\end{lstlisting}
 
 \subsubsection{Einrichtung der Datenbank}
 Die SQL Datenbank 'SQLite'  und der Object-Relationship-Mapper 'Sequelize' können einfach über den NPM dem Projekt hinzugefügt werden.  
 Nach diesem Schritt kann die Anbindung und Einpflegung folgen. Hierzu wird ein Datenbank Utility Modul angelegt, dieses soll die grundlegende Konfiguration der Datenbank enthalten und ausführen. All dies kann direkt über Sequelize erfolgen, welches im Hintergrund die notwendigen Schritte vornimmt. Es muss der Dialekt 'sqlite' angegeben werden und der Pfad unter welchem die Datenbank als Datei gespeichert werden soll. \\ 
 Gemäß dem logischen Aufbau einer SQL Datenbank folgt nun das Konfigurieren und Anlegen der Tabellen und deren Beziehung untereinander. Dieser Arbeitsschritt erfolgt relativ intuitiv. Im folgenden Code-Beispiel wird die Tabelle bzw. Sequelize Model 'student' im Modul tables konfiguriert.
 
 \begin{lstlisting}[caption=Anlegen einer Tabelle und deren Beziehungen]
exports.student = (sequelize, Sequelize) => {
	return sequelize.define('student', {
		id: {
			type: Sequelize.INTEGER,
			autoIncrement: true,
			allowNull: false,
			primaryKey: true
		},
		name: {
			type: Sequelize.STRING,
			allowNull: false
		},
	})
};
 \end{lstlisting}
 Im Datenbank Utility Modul wird nun die Konfiguration geladen und anschließend dessen Beziehung zu anderen Entitäten eingestellt. Durch Sequelize kann hier ein relativ humanes Sprachbild verwendet werden. Das folgende Beispiel zeigt die Beziehungen zwischen EduSession und Student an:
 \begin{lstlisting}[caption=Konfiguration von Entitätsbeziehungen]
 EduSession.hasMany(Student, { onDelete: 'cascade' });
 Student.belongsTo(EduSession);
 \end{lstlisting}
 Nach dem selben Schema werden nun für alle Modelle entsprechende Tabelle angelegt und deren Beziehungen untereinander festgelegt. 
 \paragraph{Datenbank Interaktion}
 Um Datenbank Anfragen (Queries) zu stellen, bietet Sequelize viele Optionen an. Diese müssen nicht in SQL geschrieben werden, sondern sind normale JavaScript Funktionen. Jedes innerhalb Sequelize definierte Model bietet diese automatisch an. Als Beispiel könnte nun über 
 \texttt{const studentToFind = Student.findByPk(1);} die Entität mit der ID 1 geladen werden. All diese bereitgestellten Funktionen sind JavaScript Promises, d.h. sie werden asynchron ausgeführt und es Bedarf entsprechendes Handeln im Fehlerfall.
 Im Erfolgsfall befindet sich in der Variabel nun das Objekt der Entität und dieses bietet wiederum Funktionen zur Interaktion an. Eine ausführliche Dokumentation findet sich auf der Webpräsenz von Sequelize. 
 Die in Tabelle \ref{tab:mvcschema} gelisteten Modelle werden gemäß der Arbeitsteilung des MVC-Schemas hauptsächlich direkt mit Sequelize arbeiten und den Controllern entsprechende Funktionen bereitstellen.
 \subsubsection{Implementierung des LehrerInnen Bereiches}\label{sec:implementlehrer}
 %hier halt sessions zum einlogge
 % Das Grundsetup wenn alles neu
 % User Anlegen / Freischalte / Sicherheit bei der Datenfreigabe 
 % Anlegen von Lehreinheiten + Brainstorming + Quizzing
 Nachdem die aus Abschnitt \ref{sec:mvc} genannten Model-Module angelegt wurden, welche direkt mit der Datenbank interagieren, müssen anschließend Controller für die verschiedenen Abschnitte des Webservers implementiert werden. Jedes Model hat dabei einen zugehörigen Controller, welcher entsprechende Funktionen für die Routen exportieren soll. Wie in Listing 3 beispielhaft zu sehen, wird die GET-Route '/student' mit der nach außen hin exportierten Funktion \texttt{getStudent} assoziiert. Um eine einheitliche Struktur zu gewährleisten, werden exportierte Funktionen eines Controller Moduls immer nach der jeweilig bedienten Request-Methode benannten (GET, POST, etc.). \\
 \paragraph{View Rendering mit PUG}
Für alle Views soll die Template Engine Pug zum Einsatz kommen (siehe auch Kapitel \ref{sec:konzept} Konzept). Dazu wird PUG für das Rendern aller nicht statischen Routen im Hauptmodul der Software registriert. PUG macht das Schreiben von HTML Dokumenten sehr komfortable und unterstützt Vererbung (Layouting). So wird zunächst für den LehrerInnen Bereich ein Main Layout angelegt, von dem später alle anderen, diesem Bereich zugehörigen, Views erben. Die Layouts können zusätzlich so genannte Blöcke enthalten, welche später dann dynamisch mit Inhalten von Views gefüllt werden, welche vom Main Layout erben. Die Controller können Daten an die Views übergeben, welche von diesen dann dargestellt werden. Dazu bietet PUG wie die meisten Template Engines Funktionen an, wie das Iterieren über Datensätze mittels Schleifen ermöglicht oder dynamisch generierte Inhalte abhängig von konditionalen Ausdrücken. Sämtliche an die Clients ausgelieferte HTML Dokumente sollen von PUG on-demand generiert werden. PUG benutzt seine eigene Templating-Language, welche sich deutlich von dem bekannten HTML unterscheidet, aber sehr intuitiv und schnell zu lernen ist. So wird die Hierarchie der HTML Elemente allein durch Einrückung bestimmt. Somit entfällt das Schließen dieser komplett. \\ 
Es werden anschließen für alle Modelle Views angelegt, damit Lehrende sich anmelden und einloggen sowie neue Benutzer anlegen und editieren können. Ebenso für das erstellen von Lehreinheiten vom Typ Brainstorming und Quiz. Dabei werden bei Dateneingabe seitens der Nutzende Formulare verwendet, welche anschließend via POST Request an den Server geschickt und dort von Controllern und ihren jeweiligen Modellen ausgewertet. Das tatsächliche Ausführen und dessen Entwicklung wird im später folgenden Abschnitt \textbf{\ref{sec:implementsessions} Implementierung der Lehreinheiten} beschrieben. 
\\ 
HIER NOCH EIN LISTING?
\paragraph{UI Design}
Aufbauend auf den Abschnitt \ref{sec:uientwurf} des Entwurf-Kapitels wird zur Designumsetzung das Frontend-CSS-Webframework \textbf{Bootstrap} genutzt. Dieses kann ebenfalls einfach über NPM dem Projekt hinzugefügt werden. Als Design-Theme soll das frei erhältliche Bootstrap Theme \texttt{Neat} von \textbf{freehtml5.co} die Grundlage bilden. Dieses wird gänzlich im Backend-Bereich des Lehrkraftzugangs genutzt, angepasste Teile anknüpfend für den Student Client und den Presenter Client, welche insbesondere noch durch eigenen CSS Code für die Nutzung auf Großbildgeräten wie Fernsehern und Projektoren optimiert wird. Es wird sichergestellt dass sich die Software angenehm auf stationären wie mobilen Endgeräten nutzen lässt. 

\begin{figure}[h!]
	\centering
	\includegraphics[width=0.9\linewidth]{bilder/screenshot_lehreinheiten}
	\caption[Screenshot UI Design Lehreinheitenbereich]{UI Design der Applikation beispielhaft illustriert durch ein Screenshot des Lehreinheiten Bereiches im Backend Zugang des Teacher Clients.}
	\label{fig:screenshot_lehreinheiten}
\end{figure}

\paragraph{Absicherung des Bereiches}
Bestimmte Bereiche der Applikation sollen nur registrierten und freigeschalteten Benutzenden zugänglich sein. Beim erstmaligen Initialisieren wird der Server mit einem Super-Administrator Account eingerichtet. Nur dieser soll neue Nutzer freischalten, andere Lehrende zu Administratoren ernennen und die Applikation auf Werkseinstellungen zurücksetzen können. Eine entsprechende Einrichtungsmaske soll beim ersten Serverstart automatisch erscheinen. Um dies technisch zu realisieren, wird das aus dem \textbf{Konzept} Kapitel genannte \texttt{express-session} NPM-Modul genutzt, welches bereits vollständig mit der von Sequelize verwalteten SQLite Datenbank einsatzbereit ist. Nach der Einrichtung im Hauptmodul wird eine eigene Middleware geschrieben, welche bei allen abgesicherten Routen als erstes aufgerufen wird und überprüft, ob die vom Client übertragene Session noch gültig ist.
\begin{lstlisting}[caption=Code der Authentifizierungs Middleware]
module.exports = (req, res, next) => {
	if(!req.session.isLoggedIn) return res.redirect('/teacher/login');
	next();
}
\end{lstlisting}
Falls dies nicht Fall ist, wird auf die Login Seite verwiesen. 
Das Verwalten der Sessions und das generieren von Cookies für die Client-Seite wird automatisch von dem Modul übernommen.
 \subsubsection{Umsetzung des Client Softwareanteile}\label{sec:implementclients}
 %Hier auch setup, browserify! 
 % vuejs, etc pp
Nachdem die Funktionalitäten Erste Initialisierung der Software, Login/Logout und Benutzerverwaltung sowie das Anlegen und Verwalten von Lehreinheiten des Typs Brainstorming und Quiz implementiert sind, sollen die Lehreinheiten auch aktiv ausgeführt werden können. Dies bildet die Hauptfunktionalität der Software und verlangt mehr Interaktion auf der Client Seite.\\ \\ Die \textbf{Browserify} Software wird einfach über den NPM der Projekt hinzugefügt und ist sofort einsatzbereit. Alle im Abschnitt \ref{sec:clientjs} des Konzeptkapitels erwähnten JavaScript Lösungen sind als NPM Packet verfügbar und werden ebenfalls dem Projekt hinzugefügt. Pro Client (Teacher Client, Student Client, Presenter Client) wird ein Development-Modul angelegt. In diesem können alle benötigten JavaScript Bibliotheken normal importiert und genutzt werden. 
Via Browserify wird anschließend pro Client ein Production Modul generiert, welches alle notwendigen Importe bündelt. Um diesen Prozess zu automatisieren, wird ein Skript erstellt, welches vom NPM ausgeführt werden kann. Nur das Production Modul muss via \texttt{script} Tag in das jeweilige Pug Template pro Client eingebunden werden. Dies reduziert gleichzeitig die Anzahl notwendiger GET-Requests auf der Client-Seite. \\
Pro Client wird die UI mit dem JavaScript Framework \textbf{Vue.js} kontrolliert und verwaltet. 
Auf dem HTML Layout wird ein \texttt{div} Element als Ankerpunkt definiert und alle unterliegenden Elemente stehen fortan zur dynamischen Anpassung bereit. Mittels Datenbindung (Data-Binding) hält VueJS die angezeigten Informationen auf dem UI aktuell. VueJS ist dabei pro Client als einfaches JavaScript Objekt auch von außen ansprechbar, was die Schnittstelle für andere Bibliotheken, insbesondere SocketIO, bildet. \textbf{SocketIO} auf der Client-Seite ist für den gesamten Datenverkehr zwischen Server und Client verantwortlich. Sowohl auf Server- wie auch Client-Seite können Listener programmiert werden, die auf bestimmte Ereignisse (Events)  von der jeweils anderen Seite ausgelöst, lauschen. Im Ereignisfall wird eine anonyme Funktion aufgerufen, welche sich um die eintreffende Daten kümmert. Die ausgetauschten Daten müssen hierbei nicht zwangsläufig zuvor indas  JSON-Format umgewandelt werden, wie dies sonst bei REST-Apis üblich ist. \\ Auf der Server-Seite kümmert sich das Modul \texttt{ioSocketHandler} um alle eingehend Web-Socket Verbindungen und ordnet diesen zunächst einem Namensraum (Namespace) zu. Ein Student Client wird dabei immer dem Namensraum für Student Clients zugeordnet und steht als Socket Objekt zur Verfügung, übliche Clients diesem Schema folgend.  Nach erfolgreicher Verbindung wird dem Student Client eine Liste verfügbarer Lehreinheiten geschickt und diesem auf der Client Seite dargestellt. Pro Lehreinheittyp (Brainstorming und Quiz) gibt es einen  'Session Handler', der als JavaScript Klasse implementiert ist. Startet eine Lehrkraft eine Lehreinheit wird abhängig vom Typ eine neuen Klasseninstanz angelegt und eine Referenz gehalten. Tritt nun eine Schülerin oder ein Schüler der Lehreinheit bei, übergibt das übergeordnete 'Socket Handler'-Modul das Socket-Objekt der Klasseninstanz der Lehreinheit. Die gesamte Logik und Kommunikation der auszuführenden Lehreinheit (Session) wird von der jeweiligen Klasse übernommen. Beendet eine Lehrkraft die Session, wird diese aus dem Speicher entfernt und steht nicht mehr zum beitreten zur Verfügung. Pro gestartete Session kann über einen speziellen Link der passende Presenter Client aufgerufen werden. Dieser wird ebenfalls über das \texttt{ioSocketHandler} Modul der jeweiligen Lehreinheiten Klasseninstanz zugeordnet. Es folgt ein Codeauszug welcher den Datenaustausch zwischen Server und Client zeigt.
\begin{lstlisting}[caption=Server Socket Event Emitierung]
/// TEACHER :::::::
updateSessionT() {
	this.socketT.emit("updateSession", this.session);
}
\end{lstlisting}
\footnotesize
Der Server schickt das Event 'updateSession' an den Teacher Client.
Als Inhalt der Nachricht wird das Session Objekt ('this.session') übermittelt.
\begin{lstlisting}[caption=Client Socket Event Listener]
// Sever tells client to update the session object
socket.on("updateSession", function (newSession) {
	console.log("getting fresh session from server...", newSession)
	if (newSession && newSession.id == vue.session.id) {
		vue.session = newSession;
	}});
\end{lstlisting}
\footnotesize
Der Teacher Client lauscht auf das Event 'updateSession'. Trifft dieses ein,
wird eine anonyme Funktion aufgerufen, welche sich dem Inhalt der Nachricht annimmt.
Diese überprüft in diesem Fall zunächst, ob sich die ID des zu aktualisierenden Session Objekts
mit dem ursprünglichen deckt. Anschließend wird das alte Session Objekt auf das neue referenziert. 

\normalsize 
 
\subsubsection{Problemstellen der Implementierung}\label{sec:probsserver}
%ioSessionHandler Socket Handling
%Word Cloud passende finden und praktikabel 
%Wifi Hotspot node module 
%Allgemeiner Umfang von VueJS und sein Einsatz
Während der Implementierung stellte sich anfangs das Verbindungsmanagement der verbundenen Clients über SocketIO als instabil heraus, da Sockets bei jedem Verbindungsabbruch sich zwar selbständig erneut verbinden, jedoch immer unter einer neuen Session ID. Dies führte zunächst zu unerwartetem Verhalten während einer ausgeführten Lehreinheiten Ausführung. Dem konnte aber durch zusätzliche Authentifizierungsdaten entgegengewirkt werden. Bei mobilen Geräten wie Smartphones scheint dies auch geräteabhängig zu sein, da manche hier die Verbindung z.B. beim Ausschalten des Bildschirms sofort unterbrechen, während andere diese im Hintergrund weiter aufrecht erhalten. \\ Ebenso war es schwierig ein gut funktionales Word-Cloud / Wörterwolke Modul zu finden, welches sich ohne nennenswerte Probleme mit VueJS im Einklang nutzen lassen konnte. Generell wird VueJS in diesem Projekt recht rudimentär eingesetzt, was seine Funktionsweise zwar nicht einschränkt, jedoch das volle Potential dieser Web-Frontend-Engine nicht gänzlich nutzt. Eine tiefgreifendere Einarbeitung in VueJS ist aus Zeitmanagement Gründen nicht erfolgt. \\ \\ Das NPM-Modul \texttt{node-hotspot} wurde zwar gemäß der Instruktionen der Entwickler implementiert, allerdings konnte auf mehreren Microsoft Windows Testsystemen nicht selbstständig ein WLAN-Hotspot aktiviert werden. Alternative Module erwiesen sich als ungenügend. 
 
 
 
\newpage

\section{Erprobung}\label{sec:erprobung}
\newpage

 \section{Fazit}\label{sec:auswertung}
In diesem vorletzten Kapitel der Arbeit wird die implementierte Software als Ergebnis mit der ursprünglichen Zielsetzung verglichen. Anschließend werden gesammelte Erfahrungen, die mit der Umsetzung des Projekts einhergingen, reflektiert. 
Im letzten Kapitel {\ref{sec:ausblick} Ausblick} wird auf die Zukunft des Projektes näher eingegangen sowie über einzelne Bestandteile des Projekts hinsichtlich Optimierung gesondert diskutiert.

\subsection{Zusammenfassung der Ergebnisse}\label{sec:umvsplan}
An Schulen bedarf es trotz der künftigen finanziellen Unterstützung durch den \emph{DigitalPakt Schule} an kostengünstigen und einfache digitalen Lösungen, damit das Geld an den richtigen Stellen sinnvoll eingesetzt werden kann.

Das Konzept einer kostengünstigen, auf Webtechnologien basierenden Softwarelösung zur Unterstützung von interaktiven Unterrichtsmethoden konnte erfolgreich umgesetzt werden.\\
 
Die entwickelte Serversoftware ist flexibel auf verschiedenen Betriebssystemen und unterschiedlicher Netzwerk-Infrastruktur einsetzbar. Dabei ist ein Offline-Verwendung im Intranet möglich und es müssen keinerlei Ressourcen aus dem Internet geladen werden. Ein Betrieb im Internet mit verschlüsselter Kommunikation ist im Bedarfsfall möglich. Diese kann mit wenigen Schritten eingerichtet werden. 
\\ \\
Der Betrieb auf dem Einplantinencomputer \emph{Raspberry Pi 3} konnte erfolgreich getestet werden und die Hardwareanforderungen an das Server-System gering gehalten werden, was einem niedrigen Anschaffungspreis zur Folge hat.  

Nutzende können die Server-Software mit wenig Aufwand installieren, es ist
lediglich die \emph{JavaScript}-Laufzeitumgebung \emph{Node.js} einzurichten. Als Client kann jeder moderne Webbrowser auf mobilen und stationären Geräten genutzt werden. 

Es konnten erfolgreich drei Webbrowser-Client Lösungen für Lehrende, Teilnehmende und Anzeigeräte in Unterrichtsräumen (Projektoren, digitale Tafeln, etc.) implementiert werden. 
Diese können je nach Situation und Anforderung auf unterschiedlichen Geräten ausgeführt werden, wobei sich die Benutzeroberfläche auf die verwendete Bildschirmgröße adaptiert. \\ 

Eine Lehrkraft kann sich registrieren, Lehreinheiten anlegen, diese ausführen und den Server grundlegend administrieren. Die zwei Unterrichtsmethoden Brainstorming und Quiz wurden erfolgreich  umgesetzt und teilnehmende Schülerinnen und Schüler können an diesen partizipieren. Anschließend besteht die Möglichkeit Ergebnisse auszuwerten. Der Server kann mehrere Nutzer und ausgeführte Unterrichtseinheiten gleichzeitig bedienen. Die Benutzeroberfläche wurde simple gehalten, damit auch weniger technisch versierte Nutzende  
die Software einsetzen können. 
Eine Lösung zur Generierung eines autarken WLANs konnte nicht zufriedenstellend gefunden werden, dafür wurden alternative Lösungen, wie das Nutzen eines Smartphones als WLAN-Zugangspunkt, ausgearbeitet. \\ 

Die Anwendung läuft stabil und neuste Spracheigenschaften der Programmiersprache \emph{JavaScript} nach Spezifikation \emph{ECMAScript} 2015, 2016 und 2017 wurden erfolgreich im Quellcode eingesetzt.

\subsection{Erfahrungsauswertung}\label{sec:erfahrungen}
Die Auseinandersetzung mit Unterrichtsmethoden und der Digitalisierung an 
Schulen war interessant und es konnte ein besseres Verständnis für damit verbundene Probleme entwickelt werden. \\ 

Durch den Implementierungsteil der Arbeit wurden vielerlei Erfahrungen hinsichtlich der Entwicklung eines verteilten Systems in Form eine Webanwendung gesammelt. Insbesondere die gewonnen Kenntnisse im Umgang mit den Frameworks \emph{Node.js}, \emph{Express} und \emph{Socket.IO} waren lehrreich und der erlangte Wissensstand kann als Basiswissen hinsichtlich vielerlei Arten von Projekten in Zukunft genutzt werden. Auf der Client Seite war der Einblick in die Arbeitsweise des Frontend-Webframework \emph{Vue.js} interessant. Das Vorwissen über die Konkurrenten \emph{Angular} und \emph{React} war hilfreich, wurde jedoch um vielerlei neue Ansätze ergänzt.
Die Arbeitsweise des \emph{WebSocket} Protokolls wurde gelernt und mit der \emph{JavaScript} Bibliothek \emph{Socket.IO} erfolgreich eingesetzt.
Die neuen Funktionen nach Spezifikation \emph{ECMAScript} 2015, 2016 und 2017 des Sprachkerns der Programmiersprache \emph{JavaScript} wurde verinnerlicht und das Wissen um die Sprache intensiviert.     

\newpage

\section{Ausblick}\label{sec:ausblick}
%electron!
%  Eine hier angesetzte Refaktorierung ist vuejs und socketIO
% REST statt Sockets
% 
\subsection{Optimierungspunkte der Software}\label{sec:opti}
Der Nachrichtenaustausch, welcher über das \emph{WebSocket} Protokoll via \emph{Socket.IO} realisiert ist, sollte mehr vereinheitlicht werden. Grundsätzlich lassen sich jede Art von \emph{JavaScript} Daten übertragen; ein strengeres Konzept kann hier das Verständnis für andere Entwickler fördern und den Code sauberer halten. Gerade der Ausführungscode der interaktiven Unterrichtsmethoden könnte noch besser gekapselt und noch sinnvoller aufgeteilt werden, auch in Hinblick auf die Daten, welche zwischen Server und Client ausgetauscht werden. Der Installationsprozess könnte ggf. noch automatisierter erfolgen und so wenig Technik affinen Nutzenden die Installation erleichtern.   
Der Code der Web-Clients, welcher im Webbrowser ausgeführt wird, kann mit mehr Kenntnissen über die Entwicklung mit \emph{Vue.js} und \emph{Socket.IO} optimiert werden. 
\subsection{Anknüpfende Ansätze}\label{sec:ansatze}
Das Projekt soll zukünftig weiterhin ausgebaut werden. Während der Entwicklung kamen immer wieder Ideen für weitere Funktionen auf, welche aber aus Gründen der Priorität nicht implementiert worden sind oder nur als Konzept vorlagen. Dies wären z.B. Funktionen, die den Dozierenden noch intensiver während des Unterrichts unterstützen oder das Schreiben einer API, um die Software an andere Systeme anbinden zu können. Zum Zeitpunkt des Abschlusses des Projekts kann eine Lehrkraft ein Lehreinheit anlegen, welche eine Unterrichtsmethode (Brainstorming oder Quiz) beinhalten kann. Die Umsetzung weiterer Unterrichtsmethoden wäre wünschenswert. Ebenso die Option, dass eine Lehreinheit mehrere Unterrichtsmethoden beinhalten kann. Ein weiteres Vorhaben wäre es, die Software in einem \emph{Fork} nach dem REST-Design aufzubauen und die Kommunikation dementsprechend umzugestalten, um anschließend zu vergleichen, welche Vorgehensweise entsprechende Vor- und Nachteile mit sich bringt. \\ Die Entwicklung einer Desktop-Applikation wäre dank der 
\emph{Node.js} Basis des Projekts mit dem Framework \emph{Electron} realisierbar.
\newpage


%\fancyhead[L]{\textsl{\small \leftmark}}
%\input{sections/07Zusammenfassung.tex}%\newpage
%\input{sections/08Ausblick.tex}\newpage

%\addcontentsline{toc}{section}{Literaturverzeichnis}
%\printbibliography[nottype=online, title={Literaturverzeichnis}]
%\printbibliography[type=online,title={Online - Bildquellen}]\newpage

\printbibliography[title={Literaturverzeichnis}]
\newpage


\addcontentsline{toc}{section}{Abbildungsverzeichnis}
\listoffigures\newpage
\addcontentsline{toc}{section}{Tabellenverzeichnis}
\listoftables

%\clearpage%\vspace*{-3cm}
%\newpage

\addcontentsline{toc}{part}{Anhang}

\fancyhead[L]{\textsl{\small \leftmark \hspace{0.8cm}\rightmark}}

%\appendix % Für Anhänge
%\input{anhang/Zemente}
%\input{anhang/Nachmessung_Zemente}



\end{document}
